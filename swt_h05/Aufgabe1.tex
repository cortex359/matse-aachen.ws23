\documentclass[main.tex]{subfiles}

\begin{document}

\section{A1 Scrum: Akzeptanzkriterien vs. Definition of Done}
Erläutern Sie mit eigenen Worten den Unterschied zwischen den Akzeptanzkriterien einer User Story und der Definition of Done bei Scrum.

\subsection{Lösung 1}

Aus \cite{openai2023chatgpt4}:
\begin{quote}

In Scrum gibt es zwei wichtige Konzepte, die oft verwechselt werden, aber unterschiedliche Zwecke erfüllen: Akzeptanzkriterien einer User Story und die Definition of Done. Hier ist eine klare Erläuterung der Unterschiede:


\renewcommand{\labelenumi}{\arabic{enumi}.}
\begin{enumerate}

\item Akzeptanzkriterien einer User Story:
\begin{enumerate}
   \item Was es ist: Akzeptanzkriterien sind spezifische Bedingungen, die eine User Story erfüllen muss, um als "erfüllt" oder "abgeschlossen" angesehen zu werden. Sie sind in der Regel vom Product Owner festgelegt und geben an, was die User Story leisten soll.
   \item Zweck: Diese Kriterien helfen dem Entwicklungsteam zu verstehen, was der Product Owner von der User Story erwartet. Sie dienen als Leitfaden, um sicherzustellen, dass die entwickelte Funktionalität den Anforderungen des Benutzers entspricht.
   \item Fokus: Akzeptanzkriterien konzentrieren sich auf die spezifische Funktionalität und das Verhalten einer einzelnen User Story. Sie sind oft benutzerspezifisch und situationsspezifisch.
\end{enumerate}
\item Definition of Done (DoD):
\begin{enumerate}
    \item Was es ist: Die Definition of Done ist eine klare Liste von Anforderungen und Bedingungen, die erfüllt sein müssen, damit jede Arbeitseinheit (nicht nur User Stories, sondern alle Arbeitsergebnisse) im Scrum-Prozess als "fertig" betrachtet wird. Dies wird meist vom gesamten Scrum-Team festgelegt.
    \item Zweck: Die DoD stellt sicher, dass alle Arbeitsergebnisse eine konsistente Qualität aufweisen und alle notwendigen Schritte wie Code-Reviews, Testing, Dokumentation usw. durchlaufen haben. Sie ist ein Qualitätsstandard für die Auslieferung.
    \item Fokus: Die Definition of Done ist generisch und gilt für alle Arbeitspakete im Projekt. Sie stellt sicher, dass unabhängig von der spezifischen User Story alle Ergebnisse einen bestimmten Qualitätsstandard erfüllen.
\end{enumerate}
\end{enumerate}

Zusammengefasst: Während die Akzeptanzkriterien einer User Story spezifizieren, was benötigt wird, damit eine spezifische Story die Bedürfnisse des Endbenutzers erfüllt, legt die Definition of Done allgemeine Qualitätsstandards und erforderliche Schritte fest, die für jede Art von Arbeit im Projekt zu erfüllen sind. Beide Konzepte sind entscheidend für die erfolgreiche Umsetzung von Scrum-Projekten.
\end{quote}


\end{document}
