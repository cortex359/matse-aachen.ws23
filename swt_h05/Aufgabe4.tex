\documentclass[main.tex]{subfiles}

\begin{document}

\section{A4 Online-Reservierungssystem für Impfungen}
Sie erheben nun die Anforderungen für das Online-Reservierungssystem für Impfungen, siehe
Aufgabe A3. Sie wollen zunächst auf fachlicher Ebene den Nachrichtenfluss zwischen Patienten
und Web-Anwendung modellieren. Sie haben bereits die folgenden Komponenten identifiziert:

\begin{itemize}
    \item Der Patient als menschlicher Akteur
    \item Die Web-Anwendung als GUI
    \item Ein Identitätsmanagement als Hintergrunddienst
    \item Ein Termin-Verwaltungssystem als Hintergrunddienst
\end{itemize}
Der reguläre Nachrichtenfluss zwischen diesen Komponenten gestaltet sich wie folgt:


\begin{itemize}
    \item Der Patient besucht die Web-Anwendung und meldet sich mit seinem Benutzernamen und Passwort an (Sie können davon ausgehen, dass der Patient bereits registriert ist).
    \item Die Webseite prüft die Gültigkeit der Anmeldung im Identitätsmanagement.
    \item Bei ungültigen Anmeldedaten wird dem Patienten durch die Web-Anwendung angeboten, sein Passwort zurückzusetzen. Dazu wird eine entsprechende Operation im Identitätsmanagement aufgerufen. Der Nachrichtenfluss endet dann.
    \item Ansonsten ruft die Web-Anwendung die möglichen Wochentage für eine Impfung vom Termin-Verwaltungssystem ab und zeigt diese an.
    \item Der Patient wählt einen der möglichen Tage aus. Die Web-Anwendung ruft dann die möglichen Uhrzeiten vom Termin-Verwaltungssystem ab und zeigt diese an.
    \item Der Patient wählt eine Uhrzeit aus und bestätigt die Buchung. Dadurch wird der Termin im Termin-Verwaltungssystem reserviert und im persönlichen Kalender im Identitätsmanagement eingetragen.
    \item Falls bei der Buchung angegeben, erhält der Patient optional eine E-Mail mit dem gebuchten Termin. Diese wird vom Identitätsmanagement gesendet.
\end{itemize}

Modellieren Sie den Nachrichtenfluss zwischen Patienten und den beteiligten Systemen durch ein
UML-Sequenzdiagramm.

\subsection{Lösung 4}

\end{document}
