\documentclass[main.tex]{subfiles}

\begin{document}

\section{A3 Klassisch oder agil?}

Eine medizinische Gemeinschaftspraxis mit mehreren Ärzten:innen und Arzthelfern:innen
möchte Sie als Softwaredienstleister beauftragen, ein Online-Reservierungssystem für Impfun-
gen zu entwickeln. Es soll ein Festpreis für die Lieferung der Software vereinbart werden. Der
Auftraggeber hat bisher kaum Erfahrung mit Softwareprojekten gemacht. Außerdem stellen Sie
schon im ersten Gespräch fest, dass es verschiedene Auffassungen über den Funktionsumfang
und die Details der fachlichen Anforderungen gibt.\\

Geben Sie an, welche Vorgehensweise Sie als Projektleitung für dieses Projekt empfehlen würden.
Diskutieren Sie die Chancen und Risiken des Einsatzes von Scrum für dieses Projekt.

\subsection{Lösung 3}
Für ein solches Softwareprojekt mit einer medizinischen Gemeinschaftspraxis, die ein Online-Reservierungssystem für Impfungen entwickeln möchte, empfehle ich folgende Vorgehensweise:

### Vorgehensweise:

1. **Anforderungsanalyse und Klärung:**
   - Führen Sie detaillierte Gespräche mit den Stakeholdern (Ärzten, Arzthelfern, ggf. Patientenvertretern) durch, um die Anforderungen genau zu verstehen und zu dokumentieren.
   - Erstellen Sie eine klare Spezifikation der Anforderungen, inklusive Funktionsumfang, Benutzeroberfläche, Datenschutzbestimmungen und anderen rechtlichen Anforderungen.

2. **Festpreisangebot mit Flexibilität:**
   - Da der Auftraggeber einen Festpreis wünscht, legen Sie diesen basierend auf der initialen Anforderungsanalyse fest. Beachten Sie jedoch, dass Änderungen im Projektverlauf wahrscheinlich sind. Planen Sie daher einen gewissen Spielraum für Anpassungen ein.

3. **Agile Methodik mit Scrum:**
   - Implementieren Sie eine agile Vorgehensweise, vorzugsweise Scrum, um Flexibilität im Entwicklungsprozess zu gewährleisten. Dies ermöglicht eine iterative Entwicklung und regelmäßige Abstimmungen mit dem Auftraggeber.

4. **Regelmäßige Kommunikation und Reviews:**
   - Planen Sie regelmäßige Meetings zur Überprüfung des Projektfortschritts und zur Diskussion von Anpassungen. Dies gewährleistet, dass das Endprodukt den Erwartungen des Auftraggebers entspricht.

5. **Prototyping und Feedback-Schleifen:**
   - Entwickeln Sie frühe Prototypen des Systems, um Feedback zu sammeln und sicherzustellen, dass die Entwicklung in die richtige Richtung geht.

6. **Dokumentation und Schulung:**
   - Dokumentieren Sie die Software umfassend und bieten Sie Schulungen für die Nutzer an, um eine effiziente Nutzung des Systems sicherzustellen.

### Chancen und Risiken des Einsatzes von Scrum:

#### Chancen:
1. **Flexibilität:** Scrum erlaubt es, auf Änderungen in den Anforderungen schnell zu reagieren, was besonders wichtig ist, da das Verständnis des Auftraggebers für das Projekt zu Beginn begrenzt sein könnte.
2. **Transparenz:** Durch regelmäßige Sprints und Reviews bleibt der Auftraggeber stets über den Fortschritt informiert.
3. **Kundenorientierung:** Direktes Feedback des Auftraggebers wird in jeder Phase des Projekts berücksichtigt, was die Kundenzufriedenheit erhöht.
4. **Qualitätssicherung:** Kontinuierliches Testing und Anpassungen während der Entwicklung fördern eine hohe Qualität des Endprodukts.

#### Risiken:
1. **Unklare Anforderungen:** Anfangs unklare oder sich ändernde Anforderungen können zu häufigen Änderungen im Projektverlauf führen.
2. **Budgetüberschreitungen:** Bei einem Festpreisvertrag können häufige Anforderungsänderungen zu Budgetproblemen führen.
3. **Auftraggeber-Einbindung:** Der Erfolg von Scrum hängt stark von der aktiven Beteiligung und dem Engagement des Auftraggebers ab.
4. **Missverständnisse:** Ohne Erfahrung im Umgang mit agilen Methoden kann es beim Auftraggeber zu Missverständnissen bezüglich des Prozesses kommen.

Insgesamt wäre Scrum eine geeignete Methode für dieses Projekt, da es Flexibilität und ständige Anpassung an die Bedürfnisse des Auftraggebers ermöglicht. Es ist jedoch wichtig, von Anfang an klare Kommunikationskanäle zu etablieren und den Auftraggeber in den Prozess einzubinden.

\end{document}
