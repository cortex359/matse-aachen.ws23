\documentclass[main.tex]{subfiles}

\begin{document}

\section{A3 Klassisch oder agil?}

Eine medizinische Gemeinschaftspraxis mit mehreren Ärzten:innen und Arzthelfern:innen
möchte Sie als Softwaredienstleister beauftragen, ein Online-Reservierungssystem für Impfungen zu entwickeln. Es soll ein Festpreis für die Lieferung der Software vereinbart werden. Der
Auftraggeber hat bisher kaum Erfahrung mit Softwareprojekten gemacht. Außerdem stellen Sie
schon im ersten Gespräch fest, dass es verschiedene Auffassungen über den Funktionsumfang
und die Details der fachlichen Anforderungen gibt.\\

Geben Sie an, welche Vorgehensweise Sie als Projektleitung für dieses Projekt empfehlen würden.
Diskutieren Sie die Chancen und Risiken des Einsatzes von Scrum für dieses Projekt.

\subsection{Lösung 3}
Aus \cite{openai2023chatgpt4}:

%\begin{quote}

   \textbf{Vorgehensweise für das Softwareprojekt}
   \renewcommand{\labelenumi}{\arabic{enumi}.}
   \begin{enumerate}
       \item \textbf{Anforderungsanalyse und Klärung:}
       \begin{itemize}
           \item Führen Sie detaillierte Gespräche mit den Stakeholdern, um die Anforderungen genau zu verstehen.
           \item Erstellen Sie eine klare Spezifikation der Anforderungen, inklusive Funktionsumfang und Datenschutzbestimmungen.
       \end{itemize}

       \item \textbf{Festpreisangebot mit Flexibilität:}
       \begin{itemize}
           \item Legen Sie einen Festpreis basierend auf der initialen Anforderungsanalyse fest, aber planen Sie Spielraum für Anpassungen ein.
       \end{itemize}

       \item \textbf{Agile Methodik mit Scrum:}
       \begin{itemize}
           \item Implementieren Sie eine agile Vorgehensweise, um Flexibilität im Entwicklungsprozess zu gewährleisten.
       \end{itemize}

       \item \textbf{Regelmäßige Kommunikation und Reviews:}
       \begin{itemize}
           \item Planen Sie regelmäßige Meetings zur Überprüfung des Projektfortschritts.
       \end{itemize}

       \item \textbf{Prototyping und Feedback-Schleifen:}
       \begin{itemize}
           \item Entwickeln Sie frühe Prototypen des Systems, um Feedback zu sammeln.
       \end{itemize}

       \item \textbf{Dokumentation und Schulung:}
       \begin{itemize}
           \item Dokumentieren Sie die Software und bieten Sie Schulungen für die Nutzer an.
       \end{itemize}
   \end{enumerate}

   \textbf{Chancen und Risiken des Einsatzes von Scrum}

   \textbf{Chancen:}
   \begin{enumerate}
       \item \textbf{Flexibilität:} Scrum ermöglicht schnelle Reaktionen auf Änderungen in den Anforderungen.
       \item \textbf{Transparenz:} Regelmäßige Sprints und Reviews halten den Auftraggeber stets informiert.
       \item \textbf{Kundenorientierung:} Direktes Feedback des Auftraggebers wird berücksichtigt, was die Kundenzufriedenheit erhöht.
       \item \textbf{Qualitätssicherung:} Kontinuierliches Testing und Anpassungen fördern eine hohe Produktqualität.
   \end{enumerate}

   \textbf{Risiken:}
   \begin{enumerate}
       \item \textbf{Unklare Anforderungen:} Häufige Änderungen können zu Projektschwierigkeiten führen.
       \item \textbf{Budgetüberschreitungen:} Häufige Anforderungsänderungen können das Budget belasten.
       \item \textbf{Auftraggeber-Einbindung:} Erfolg hängt von der Beteiligung des Auftraggebers ab.
       \item \textbf{Missverständnisse:} Ohne Erfahrung können Missverständnisse bezüglich Scrum auftreten.
   \end{enumerate}
%\end{quote}

\end{document}
