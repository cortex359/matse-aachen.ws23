\documentclass[main.tex]{subfiles}

\begin{document}

\section{A5 Auto starten}
Im Zip-Archiv »sequence.zip« finden Sie eine Java-Konsolenanwendung für den Start eines Autos, siehe ILIAS. Die Anwendung besteht aus den Klassen App, Battery, Car, Engine und OilSensor.\\

Modellieren Sie, ausgehend vom Aufruf der Methode main in der Klasse App, den Nachrichtenfluss der Anwendung als UML-Sequenzdiagramm. Berücksichtigen Sie dabei folgende Hinweise:

\begin{itemize}
    \item Der Aufruf von main soll als »gefundene Nachricht« ohne Parameter modelliert werden.
    \item Die korrekten, verschachtelten Sequenznummern der Aufrufe sollen angegeben werden.
    \item Alle Datentypen, Parameter und Rückgaben sollen angegeben werden.
    \item Antwort-Nachrichten können weggelassen werden, soweit sie sich eindeutig aus den Aufrufen ergeben.
    \item Die Konsolenausgabe der App als Reaktion auf die Rückgabe von myCar.prepareStart() brauchen Sie nicht zu modellieren.
\end{itemize}

\subsection{Lösung 5}

\end{document}
