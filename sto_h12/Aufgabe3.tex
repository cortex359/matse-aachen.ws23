\documentclass[main.tex]{subfiles}

\begin{document}

\section{Aufgabe 3}
Bei der Bestimmung des Geburtsgewichts von $100$ Mädchen ergaben sich folgende gerundeten Werte:
\begin{center}
    \begin{tabular}{c|c|c|c|c|c|c|c|c|c|c}
        Gewicht in kg & $2,7$ & $2,8$ & $2,9$ & $3,0$ & $3,1$
            & $3,2$ & $3,3$ & $3,4$ & $3,5$ & $3,6$ \\ \hline
        Anzahl der Mädchen & $6$ & $8$ & $11$ & $13$ & $14$
            & $11$ & $13$ & $8$ & $9$ & $7$
    \end{tabular}
\end{center}
Testen Sie zum Niveau $\alpha = 0,05$ die Hypothese das Geburtsgewicht folgt einer
\begin{enumerate}
    \item Gleichverteilung in $[2,65;\ 3,65]$ mit der Klasseneinteilung $$
    [2,65;\ 3,0],\ (3,0;\ 3,3],\ (3,3;\ 3,65].
    $$
    \item Normalverteilung mit der Klasseneinteilung $$
    (-\infty;\ 2,8],\ (2,8;\ 3,0],\ (3,0;\ 3,2],\ (3,2;\ 3,4]\ \text{und}\ (3,4;\ \infty).
    $$
\end{enumerate}

\subsection{Lösung 3a}
Bei einer Gleichverteilung der $n=100$ Datenpunkte im Intervall $[2,65;\ 3,65]$ ist die erwartete Häufigkeit $E_i$ der Klasse $A_i$ gleich dem Produkt aus $n$ und der Klassenbreite.

\begin{center}
    \begin{tabular}{c|r|c|c}
        Klasse & Intervall      & beobachtete Häuf. & erwartete Häuf.\\
        $A_i$  &                & $O_i$                  & $E_i$ \\
        \hline
        $A_1$  & $[2,65;\ 3,0]$ & 38                     & 35 \\
        $A_2$  & $(3,0;\ 3,3]$  & 38                     & 30 \\
        $A_3$  & $(3,3;\ 3,65]$ & 24                     & 35 \\
    \end{tabular}
\end{center}

Da für $i \in \set{1, 2, 3}$ die erwartete Häufigkeit $E_i > 5$ ist, ist der Chi-Quadrat-Anpassungstest für die Untersuchung geeignet.\\

Wir berechnen die Chi-Quadrat-Teststatistik $D$:
$$\begin{aligned}
	D &= \sum_{i=1}^{d} \frac{(O_i - E_i)^2}{E_i} \\[2mm]
	&= \frac{(38 - 35)^2}{35} + \frac{(38 - 30)^2}{30} + \frac{(24 - 35)^2}{35}\\[2mm]
    &= \frac{614}{105}\\[2mm]
    &= 5,8\overline{476190}
\end{aligned}$$

Da $D \approx 5,848 < 5,991 = \chi^2_{2, 0,95}$ ist, gibt es keine ausreichende Evidenz, um die Nullhypothese einer Gleichverteilung abzulehnen. Das bedeutet, dass die Daten nicht signifikant von einer Gleichverteilung in $[2,65;\ 3,65]$ abweichen.

\subsection{Lösung 3b}
Für den Vergleich mit einer Normalverteilung der Datenpunkte werden zunächst der Erwartungswert $\overline{X}$ und die empirische Varianz $s^2$ der Stichprobe benötigt.\\

$$\begin{aligned}
    \overline{X} &= \frac{1}{n} \sum_{i=1}^{10} h_i \cdot A_i \\[2mm]
    &= \frac{1}{100} \left(
        6  \cdot 2,7 +
        8  \cdot 2,8 +
        11 \cdot 2,9 +
        13 \cdot 3,0 +
        14 \cdot 3,1 +
        \right. \\
    &\quad \left.
        11 \cdot 3,2 +
        13 \cdot 3,3 +
        8  \cdot 3,4 +
        9  \cdot 3,5 +
        7  \cdot 3,6
    \right) \\[2mm]
    &= 3,149\\[4mm]
    s^2 &= \frac{1}{n-1} \cdot \left( \sum^n_{i=1} \left(X_i^2\right) - n\cdot \overline{X}^2 \right) \\[2mm]
    &= \frac{1}{99} \cdot \left( 998,17 - 991,6201 \right)\\[2mm]
    &\approx 0,0661606
\end{aligned}$$

Für $X=\set{\text{Geburtsgewicht}}\sim \mathcal{N}(3,149;\ 0,0661606)$ werden folgende Häufigkeiten erwartet.

\begin{center}
\begin{tabular}{c|r|r|r}
Klasse & Intervall            & beob. Häuf. & erw. Häuf.\\
$A_i$  &                      & $O_i$             & $E_i$ \\\hline
$A_1$  & $(-\infty;\ 2,8   ]$ & 14                &  8,7417 \\
$A_2$  & $(    2,8;\ 3,0   ]$ & 24                & 19,3784 \\
$A_3$  & $(    3,0;\ 3,2   ]$ & 25                & 29,7384 \\
$A_4$  & $(    3,2;\ 3,4   ]$ & 21                & 25,6840 \\
$A_5$  & $(    3,4;\ \infty)$ & 16                & 16,4574 \\
\end{tabular}
\end{center}

Wobei allgemein für das Intervall $A_i = (a; b]$ die erwartete Häufigkeit wie folgt berechnet werden kann:
$$
    E_i = n \cdot \int_{a}^{b} \frac{1}{s\cdot \sqrt{2\pi}} \cdot \exp\left(-\frac{(x-\overline{X})^2}{2\cdot s^2} \right) \dx{x}
$$
ist.

Wir berechnen die Chi-Quadrat-Teststatistik $D$:
$$\begin{aligned}
    D &= \sum_{i=1}^{d} \frac{(O_i - E_i)^2}{E_i} \\[2mm]
    &\approx 5,88708
\end{aligned}$$

Da $D = 5,887 < 9,488 = \chi^2_{4, 0,95}$ gibt es keinen ausreichenden Grund, die Nullhypothese zu verwerfen. Das bedeutet es gibt keine signifikanten Beweise dafür, dass das Geburtsgewicht der Mädchen nicht normalverteilt ist.

\end{document}
