\documentclass[main.tex]{subfiles}

\begin{document}

\section{Aufgabe 3}
Bei der Bestimmung des Geburtsgewichts von $100$ Mädchen ergaben sich folgende gerundeten Werte:
\begin{center}
	\begin{tabular}{c|c|c|c|c|c|c|c|c|c|c}
		Gewicht in kg & $2,7$ & $2,8$ & $2,9$ & $3,0$ & $3,1$
			& $3,2$ & $3,3$ & $3,4$ & $3,5$ & $3,6$ \\ \hline
		Anzahl der Mädchen & $6$ & $8$ & $11$ & $13$ & $14$
			& $11$ & $13$ & $8$ & $9$ & $7$
	\end{tabular}
\end{center}
Testen Sie zum Niveau $\alpha = 0,05$ die Hypothese das Geburtsgewicht folgt einer
\begin{enumerate}
	\item Gleichverteilung in $[2,65; 3,65]$ mit der Klasseneinteilung $$
	[2,65; 3,0],\ (3,0; 3,3],\ (3,3; 3,65].
	$$
	\item Normalverteilung mit der Klasseneinteilung $$
	(-\infty; 2,8],\ (2,8; 3,0],\ (3,0; 3,2],\ (3,2; 3,4]\ \text{und}\ (3,4; \infty).
	$$
\end{enumerate}

\subsection{Lösung 3}

\end{document}
