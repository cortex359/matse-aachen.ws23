\documentclass[main.tex]{subfiles}

\begin{document}

\section{Aufgabe 1}
$384$ zufällig ausgewählte Personen wurden nach ihrem Unfall in einer bestimmten Angelegenheit befragt. Zur statistischen Auswertung wurden die Urteile jeweils in eine von $6$ Kategorien eingeordnet und in folgender Tabelle dargestellt:
\begin{center}
\begin{tabular}{c|c|c|c|c|c|c}
	Kategorie & $I$ & $II$ & $III$ & $IV$ & $V$ & $VI$ \\ \hline
	Anzahl der Urteile & $58$ & $61$ & $72$ & $67$ & $57$ & $69$
\end{tabular}
\end{center}
Testen Sie mit einem geeigneten Testverfahren zum Niveau $\alpha = 0,05$, ob in der Grundgesamtheit alle sechs Kategorien gleich wahrscheinlich sind.

\subsection{Lösung 1 Chi-Quadrat-Anpassungstest}
Der Chi-Quadrat-Anpassungstest vergleicht die beobachteten Häufigkeiten $O_i$ in den Klassen mit den erwarteten Häufigkeiten $E_i$ unter der Annahme, dass die Zahlen gleichverteilt sind.\\

Bei einer Gleichverteilung der Urteile über die $d=6$ Kategorien, erwarten wir, dass jede Klasse etwa $E_i = \frac{n}{d} = \frac{384}{6} = 64$ Urteile enthält.\\

Die Chi-Quadrat-Teststatistik $D$ wird wie folgt berechnet:
$$\begin{aligned}
	D &= \sum_{i=1}^{d} \frac{(O_i - E_i)^2}{E_i} \\
	&= \sum_{i=1}^6 \frac{(O_i - 64)^2}{64} \\
	&= \frac{1}{64} \left( 6^2 +3^2 + 8^2 + 3^2 + 7^2 + 5^2 \right)\\
	&= 3
\end{aligned}$$

Diesen Wert vergleichen wir mit dem 0,95-Quantil der Chi-Quadrat-Verteilung für $d-1=5$ Freiheitsgrade.\\

Da $D = 3 < 11,07 = \chi_{5;\ 0,95}$ können wir die Nullhypothese nicht ablehnen.


\end{document}
