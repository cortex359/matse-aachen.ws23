\documentclass[main.tex]{subfiles}

\begin{document}

\section{Aufgabe 2}
Bei einer Population von $30$ Versuchstieren wird an einem bestimmten Tag das Gewicht (in kg) gemessen. Dabei ergaben sich die folgenden Messungen:
\begin{center}
\begin{tabular}{cccccccccc}
$12,16$ & $11,53$ & $14,02$ & $11,85$ & $10,94$ & $11,83$ & $12,94$ & $11,46$ & $13,15$ & $12,70$ \\
$10,88$ & $13,24$ & $14,04$ & $10,95$ & $14,78$ & $12,39$ & $13,69$ & $11,82$ & $14,28$ & $12,96$ \\
$13,24$ & $13,42$ & $12,23$ & $15,04$ & $11,34$ & $12,28$ & $13,42$ & $13,93$ & $14,73$ & $11,28$
\end{tabular}
\end{center}
\begin{enumerate}
\item Erstellen Sie zur Übersicht der Verteilung eine Tabelle mit der Klasseneinteilung $[10,0;11,5)$, $[11,5;13,0)$, $[13,0;14,0)$, $[14,0;16,0)$. Geben Sie die absolute und relative Klassenhäufigkeit sowie die Werte für die empirische Verteilungsfunktion an. 
\item Zeichnen Sie 
\begin{enumerate}
\item das zugehörige Histogramm und 
\item die empirische Verteilungsfunktion.
\end{enumerate}
\item Berechnen Sie aus den klassierten Daten
\begin{enumerate}
\item das arithmetische Mittel
\item den Median  
\item die Modalklasse
\item das 90\% -Quantil
\item das untere Quartil 
\item die empirische Varianz sowie die empirische Standardabweichung
\end{enumerate}
\item Geben Sie den Variationskoeffizienten an.
\end{enumerate}

\subsection{Lösung 2}

\end{document}
