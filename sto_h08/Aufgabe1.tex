\documentclass[main.tex]{subfiles}

\begin{document}

\section{Aufgabe 1}
Bei einer Klassenarbeit erhielten die $25$ Schülerinnen und Schüler einer Klasse in alphabetischer Reihenfolge die Zensuren
\begin{center}
$3$, $5$, $4$, $3$, $2$, $3$, $4$, $6$, $1$, $2$, $3$, $3$, $4$, $5$, $2$, $1$, $3$, $4$,	$2$, $4$, $3$, $1$, $2$, $3$, $4$
\end{center}
\begin{enumerate}
\item Erstellen Sie eine Tabelle mit Strichliste sowie absoluter und relativer Häufigkeit jeder Zensur. Zeichnen Sie ein Stabdiagramm der empirischen Häufigkeitsverteilung.
\item Ergänzen Sie die Tabelle um die Werte für die empirische Verteilungsfunktion und zeichnen Sie diese. 
\item Berechnen Sie
\begin{enumerate}
\item das arithmetische Mittel
\item den Median 
\item den Modalwert
\item das $10\%$-Quantil 
\item das obere Quartil
\item die empirische Varianz und die empirische Standardabweichung
\end{enumerate} 
\item Geben Sie den Variationskoeffizienten an.
\end{enumerate}

\subsection{Lösung 1}

\begin{center}    
    \begin{tabular}{c|l|c|r}
        Zensur $A_j$ & Striche & Ereignisse $h_j$ & relative  Häufigkeit $r_j$ &  \\\hline
        1 & \StrokeThree            & 3 & 0,12 \\
        2 & \StrokeTwo              & 5 & 0,2  \\
        3 & \StrokeFive\StrokeThree & 8 & 0,32 \\
        4 & \StrokeFive\StrokeOne   & 6 & 0,24 \\
        5 & \StrokeTwo              & 2 & 0,08 \\
        6 & \StrokeOne              & 1 & 0,04 \\\hline
        $\sum$ & \StrokeFive\StrokeFive\StrokeFive\StrokeFive\StrokeFive & 25 & 1
    \end{tabular}
\end{center}

\end{document}
