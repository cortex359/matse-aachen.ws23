\documentclass[main.tex]{subfiles}

\begin{document}

\section{Aufgabe 5}
Messungen des systolischen Blutdrucks bei $n = 10$ Personen ergaben folgende Werte in mmHg:
$$ 124, 145, 112, 124, 136, 129, 125, 131, 142, 114 $$
Unter der Annahme, dass der Blutdruck normalverteilt ist, bestimmen Sie jeweils zum Niveau $90\%$
\begin{enumerate}
	\item ein einseitig nach oben begrenztes Konfidenzintervall für den Erwartungswert $\mu$.
	\item ein zweiseitiges Konfidenzintervall für die Standardabweichung $\sigma$.
\end{enumerate}

\subsection{Lösung 5}

Da die Standardabweichung der Grundgesamtheit nicht bekannt ist und die Stichprobengröße $n = 10$ relativ klein ist, verwenden wir das t-Verteilungsbasierte Konfidenzintervall. Die Formel für ein einseitiges Konfidenzintervall ist:

$$
	\mu \leq \overline{x} + t_{n-1,\ \alpha} \cdot \frac{s}{\sqrt{n}}
$$


Das arithmetisches Mittel der Stichprobe ist $\overline{x} = 128,2$ mmHg und die Standardabweichung der Stichprobe ist $s \approx 10,809$ mmHg.
Der t-Wert aus der t-Verteilung für das Konfidenzniveau 90 \% und $n-1 = 9$ Freiheitsgraden ist $t_{9,\ 0,9} \approx 1,3830287$.

Wir erhalten somit das einseitige Konfidenzintervall $\mu \leq 132,92754$.
Das bedeutet, wir können mit einer Sicherheit von 90\% sagen, dass der wahre Erwartungswert des systolischen Blutdrucks maximal $132,93$ mmHg beträgt.\\

Bei dem 90\%-Konfidenzintervall für die Standardabweichung $\sigma$ bestimmen wir die zwei Chi-Quadrat-Wert für $n-1$ Freiheitsgrade.
Zum einen $\chi^2_{n-1;\ \alpha/2}$ für das $\alpha / 2$-Quantil, und $\chi^2_{n-1;\ 1-\alpha/2}$ für das obere $(1-\alpha /2)$-Quantil.

Das Konfidenzintervall für $\sigma$ wird berechnet, indem die Wurzel aus der skalierten Stichprobenvarianz genommen wird, wobei die Skalierung durch die Chi-Quadrat-Werte bestimmt wird. Das Intervall ist dann
$$
	\left[
		\sqrt{(n-1) \cdot \frac{s^2}{\chi^2_{n-1;\ \alpha/2}}};\
		\sqrt{(n-1) \cdot \frac{s^2}{\chi^2_{n-1;\ 1-\alpha/2}}}
	\right]
$$

also ungefähr
$$
7,8838478 \leq \sigma \leq 17,7836988.
$$


\end{document}
