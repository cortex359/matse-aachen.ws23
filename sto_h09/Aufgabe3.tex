\documentclass[main.tex]{subfiles}

\begin{document}

\section{Aufgabe 3}
Das Umweltreferat einer Großstadt will Aufschluss darüber gewinnen, wie viele Asbestfasern pro Kubikmeter Luft im Freien in ca. einem Meter Abstand von asbestzementhaltigen Gebäudeteilen zu erwarten sind. Bei $n=14$ diesbezüglichen Messungen traten die Werte
\begin{center}
\begin{tabular}{*{7}{c}}
$980$ & $1340$ & $610$ & $750$ & $880$ & $1250$ & $2410$ \\
$1100$ & $470$ & $1040$ & $910$ & $1860$ & $730$ & $820$
\end{tabular}
\end{center}
auf, die als Ergebnisse unabhängiger normalverteilter Stichprobenvariablen angesehen werden.
\begin{enumerate}
\item Führen Sie für den Erwartungswert $\mu$ der Anzahl $X$ der unter den obigen Bedingungen vorhandenen Asbestfasern eine Intervallschätzung zum Konfidenzniveau $0,95$ durch.
\item Wie müsste das Konfindezniveau gewählt sein, damit die Länge des entstehenden Schätzintervalls gleich $500$ ist?
\end{enumerate}

\subsection{Lösung 3}

Um eine Intervallschätzung für den Erwartungswert $\mu$ der Anzahl $X$ der Asbestfasern durchzuführen, berechnen wir zunächst den Mittelwert $\bar{x}$ und die Standardabweichung der Stichprobe.
Der Mittelwert beträgt $\bar{x} \approx 1082,142857$ Fasern/m$^3$ und die Standardabweichung $s \approx 514,903704$ Fasern/m$^3$.

Sodann bestimmen wir den kritischen $t$-Wert für das gegebene Konfidenzniveau von $1-\frac{\alpha}{2} = 0,975$ und $n-1 = 13$ Freiheitsgraden und wenden die Formel für das Konfidenzintervall an:
$$
   \bar{x} \pm t_{\text{kritisch}} \cdot \left( \frac{s}{\sqrt{n}} \right)
$$

Mit $t_{\text{kritisch}} \approx 2,1603687$ erhalten wir also ein Konfidenzintervall von $784,8463$ bis $1379,4394$ Fasern/m$^3$.

Das bedeutet, dass mit einer Sicherheit von 95\% der wahre Mittelwert der Anzahl der Asbestfasern pro Kubikmeter Luft innerhalb dieses Intervalls liegt.

Um herauszufinden, wie das Konfidenzintervall gewählt werden muss, damit die Länge des Schätzintervalls genau 500 beträgt, suchen wir den $t$-Wert, der diese Gleichung erfüllt:
\begin{equiveqs}
       & 2 \cdot t_{\text{benötigt}} \cdot \left( \frac{s}{\sqrt{n}} \right) &= 500 \\
\equiv & t_{\text{benötigt}} &= 250 \cdot \frac{\sqrt{n}}{s} \\
\equiv & t_{\text{benötigt}} &= 250 \cdot \frac{\sqrt{14}}{514,903704} \\[4mm]
\equiv & t_{\text{benötigt}} &= 1,816678 \\
\end{equiveqs}

Dies entspricht einem benötigten Kofidenzintervall $\approx 9,23848\%$.

Das bedeutet, dass, um ein Konfidenzintervall mit einer Länge von 500 zu erhalten, das Konfidenzniveau auf etwa 9,24\% gesetzt werden muss.

\end{document}
