\documentclass[main.tex]{subfiles}

\begin{document}

\section{Aufgabe 3}
Das Umweltreferat einer Großstadt will Aufschluss darüber gewinnen, wie viele Asbestfasern pro Kubikmeter Luft im Freien in ca. einem Meter Abstand von asbestzementhaltigen Gebäudeteilen zu erwarten sind. Bei $n=14$ diesbezüglichen Messungen traten die Werte
\begin{center}
\begin{tabular}{*{7}{c}}
$980$ & $1340$ & $610$ & $750$ & $880$ & $1250$ & $2410$ \\
$1100$ & $470$ & $1040$ & $910$ & $1860$ & $730$ & $820$
\end{tabular}
\end{center}
auf, die als Ergebnisse unabhängiger normalverteilter Stichprobenvariablen angesehen werden.
\begin{enumerate}
\item Führen Sie für den Erwartungswert $\mu$ der Anzahl $X$ der unter den obigen Bedingungen vorhandenen Asbestfasern eine Intervallschätzung zum Konfidenzniveau $0,95$ durch.
\item Wie müsste das Konfindezniveau gewählt sein, damit die Länge des entstehenden Schätzintervalls gleich $500$ ist?
\end{enumerate}

\subsection{Lösung 3}

\end{document}
