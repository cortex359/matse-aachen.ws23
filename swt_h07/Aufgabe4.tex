\documentclass[main.tex]{subfiles}

\begin{document}

\section{A4 Maven \& JUnit: Euklidischer Algorithmus}

Erstellen Sie ein neues Maven-Projekt in Ihrer Java-Entwicklungsumgebung und implementieren Sie den modernen, iterativen Euklidischen Algorithmus. Beachten Sie, dass der Euklidische Algorithmus die mathematische Modulo-Operation nutzt, während in Java die symmetrische Modulo-Operation implementiert ist. Der Unterschied ist für negative Zahlen relevant.

Erstellen Sie anschließend in JUnit Komponententests, um stichprobenartig die folgenden mathematischen Gesetze für den größten gemeinsamen Teiler zu prüfen. Nutzen Sie dazu möglichst parametrisierte Tests.

Zu prüfende Gesetze: Seien $n, m, r$ ganze Zahlen und $\abs{n}$ der Betrag (Absolutwert) von $n$.
\begin{enumerate}
    \item $\text{ggT}(0, n) = \text{ggT}(n, 0) = \abs{n}$
    \item $\text{ggT}(1, n) = \text{ggT}(n, 1) = 1$
    \item $\text{ggT}(n, m) = \text{ggT}(m, n)$
    \item $\text{ggT}(n, n) = \abs{n}$
    \item $\text{ggT}(n \cdot r, m \cdot r) = \text{ggT}(n, m) \cdot \abs{r}$
\end{enumerate}

\subsection{Lösung 4}
Siehe Anhang.

\end{document}
