\documentclass[main.tex]{subfiles}

\begin{document}

\section{A3 Nutzwertanalyse}

Sie wollen herausfinden, ob es für Sie von Nutzen sein könnte, in die unmittelbare Nähe Ihrer Hochschule umzuziehen. Nutzen Sie eine Nutzwertanalyse, um Ihren persönlichen Wirtschaftlichkeitskoeffizienten zu diesem Vorhaben zu berechnen. Sollten Sie bereits in Hochschulnähe wohnen, dann ersetzen Sie das potenzielle Ziel durch »Arbeitsplatz« oder Großstadt«.

\begin{enumerate}
\item Legen Sie für jedes der sechs Bewertungskriterien in Tabelle 1 Ihre persönliche Gewichtung fest: $1$ für weniger wichtig, $2$ für wichtig, $3$ für sehr wichtig.
\item Legen Sie für jedes Bewertungskriterien Ihre persönliche Einschätzung zwischen $-3$ und $+3$ gemäß der Skala in Tabelle 2 fest.
\item Berechnen Sie den Wirtschaftlichkeitskoeffizienten und tragen Sie dessen Wert auf der Skala ein. Würde sich ein Umzug für Sie lohnen?
\end{enumerate}

\subsection{Lösung 3}

\begin{tabular}{|l|r|r|r|} \hline
Kriterium $K_i$                     & Gewichtung $w_i$ & Einschätzung $v_i$ & Produkt \\
\hline
Reisezeiten                         &                2 &                 +3 &   6 \\
Kosten                              &                3 &                 -2 &  -6 \\
Freizeitwert                        &                1 &                  0 &   0 \\
Kontakt zu Kommiliton:inn:en        &                1 &                 +2 &   2 \\
Konzentration auf das Studium       &                2 &                  1 &   2 \\
Identifikation mit der Hochschule   &                1 &                  0 &   0 \\
Entfernung zu Lebensgefährtin       &                3 &                 -3 &  -9 \\
\hline
Summe                               &               13 &                    &  -5 \\
\hline
\end{tabular}\\

Damit ist der Wirtschaftlichkeitskoeffizient $v = -\frac{5}{13} \approxeq -0,38$ und von der Umsetzung des Vorhabens ist abzuraten, da es eine Verschlechterung darstellen würde.

\end{document}
