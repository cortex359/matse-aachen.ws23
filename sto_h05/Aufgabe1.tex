\documentclass[main.tex]{subfiles}

\begin{document}

\section{Aufgabe 1}
Gegeben sei die folgende zweidimensionale Wahrscheinlichkeitsfunktion
%\begin{center} $f(x,y)=P(X=x,Y=y)$ \end{center}
\begin{center}
\begin{tabular}{|lcr|c|c|c|}\hline
\quad &	& $Y$ & $0$ & $1$ & \qquad \qquad \\
& $X$ & & & & \\\hline
& $0$ &	& $0/32$ & $2/32$ &  \\\hline
& $1$ &	& $1/32$ & $1/32$ & \\\hline
& $2$ &	& $3/32$ & $7/32$ & \\\hline
& $3$ &	& $10/32$ &	$8/32$ & \\\hline
& & & & & \\ \hline
\end{tabular}
\end{center}
\begin{enumerate}
\item	Berechnen Sie die Randwahrscheinlichkeiten für beide Zufallsvariablen.
\item	Ermitteln Sie die Wahrscheinlichkeitsverteilung für die Zufallsvariable $X$ unter der Bedingung $Y=1$.
%Die bedingte Wahrscheinlichkeit im mehrdimensionalen Fall lässt sich aus der allgemeinen Definition ableiten.
\item	Überprüfen Sie, ob die Zufallsvariablen $X$ und $Y$ stochastisch unabhängig sind.
\item	Berechnen Sie aus den Randverteilungen
\begin{enumerate}
\item die Erwartungswerte und
\item die Varianzen
\end{enumerate}
für $X$ und $Y$.
\item	Berechnen Sie für das gegebene Beispiel die Kovarianz $Cov(X,Y)$.
\item	Berechnen Sie den Korrelationskoeffizient $\rho_{XY}$ und treffen Sie eine Aussage über die Stärke des linearen
Zusammenhanges zwischen $X$ und	$Y$.
\end{enumerate}

\subsection{Lösung 1}

\end{document}
