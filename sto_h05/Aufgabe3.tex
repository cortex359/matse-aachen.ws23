\documentclass[main.tex]{subfiles}

\begin{document}

\section{Aufgabe 3}
Die Anzahl $X$ der abgesetzten Notebooks in einer beliebigen Woche in einer Filiale der PC-Kette Hypercom lässt sich durch eine Poissionverteilung mit Erwartungswert $4$ beschreiben.
\begin{enumerate}
\item Bestimmen Sie für eine beliebige Woche die Wahrscheinlichkeit, dass
\begin{enumerate}
\item kein Gerät
\item mindestens ein Gerät
\end{enumerate}
verkauft wird.
\item Wie groß ist die Varianz von $X$?
\item Bestimmen Sie für den Zeitraum von zwei Wochen die Wahrscheinlichkeit, dass mehr als sechs aber höchstens acht Geräte verkauft werden.
\end{enumerate}

\subsection{Lösung 3}
Die Wahrscheinlichkeitsfunktion einer Poisson-Verteilung $X\sim \text{Poi}(\lambda)$ ist gegeben durch $$
P(X = k) = \frac{{\lambda^k \e^{-\lambda}}}{{k!}}
$$ und beschreibt die Wahrscheinlichkeit, dass genau $k$ Ereignisse auftreten.

\subsubsection{a)}
Die Wahrscheinlichkeit, dass in einer beliebigen Woche kein Gerät gekauft wird $P(X{=}0)=\frac{4^0 \e^{-4}}{0!}= \e^{-4}$ also ungefähr $1,83\%$.\\

Die Wahrscheinlichkeit, dass mindestens ein Gerät gekauft wird $P(X\geq 1)$ ist $1-P(X{=}0) = 1 - \e^{-4}$ also ungefähr $98,168 \%$.

\subsubsection{b)}
Für $X\sim \text{Poi}(\lambda)$ gilt $E(X) = \Var(X) = \lambda$, was in diesem Fall $\Var(X) = 4$ ist.

\subsubsection{c)}
Sei $Y = \set{\text{\# Notbookverkäufe in zwei Wochen}}$. Der Erwartungswert $E(Y) = 2\cdot E(X) = 8$, was nach dem Additionsgesetz $$
    X + Y \sim \text{Poi}(\lambda + \mu)
$$ für zwei voneinander stochastisch unabhängige Wochen plausibel erscheint, sodass wir folgern können
$$\begin{aligned}
    P\left(Y{=}k\right) = \frac{{8^k \e^{-8}}}{{k!}}.
\end{aligned}$$
$$\begin{aligned}
    P\left(8\geq Y >6\right) &= P(Y{=}7) + P(Y{=}8)\\[2mm]
    &= \frac{8^7 \e^{-8}}{7!} + \frac{8^8 \e^{-8}}{8!} \\[2mm]
    &= \frac{262.144}{315} \cdot \e^{-8} \\[2mm]
    &\approx 27,9173 \%
\end{aligned}$$


\end{document}
