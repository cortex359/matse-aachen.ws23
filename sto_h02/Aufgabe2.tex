\documentclass[main.tex]{subfiles}

\begin{document}

\section{Aufgabe 2}
Eine Spieler bzw. eine Spielerin wirft zwei verschiedenfarbige \glqq faire\grqq~Würfel. Sind die Augenzahlen gleich und gerade, erhält sie/er einen Gewinn von 6~Euro. Sind die Augenzahlen aber gleich und ungerade, muss der Spieler bzw. die Spielerin 6~Euro an die Bank zahlen. Sind die Augenzahlen beider Würfel ungleich und stellt deren Summe eine ungerade Zahl dar, verliert der Spieler bzw. die Spielerin 3~Euro. Ansonsten (ungleiche Augenzahlen bei gerader Summe) beträgt der Gewinn 3~Euro. Wählen Sie eine zweckmäßige Zufallsvariable $X$ aus.
\begin{enumerate}
    \item Bestimmen Sie die Wahrscheinlichkeit für die Werte der Zufallsvariable.
    \item Berechnen Sie anschließend den Erwartungswert für $X$.
\end{enumerate}

\subsection{Lösung 2}
Aus dem Grundraum $\Omega = \set*{1,\ldots, 6} \times \set*{1,\ldots, 6}$ sei $A=\set*{(\omega_1, \omega_2)\in \Omega^2 \middle| \omega_1 = \omega_2}$ und $B=\set*{(\omega_1, \omega_2)\in \Omega^2 \middle| (\omega_1 + \omega_2) \text{ ist gerade}}$. \textbf{Achtung:} Dabei sind die Augenzahlen jeweils einzeln o.B.d.A. gerade, wenn ihre Summe ungerade ist.

Die Zufallsvariable $X(\omega)$ soll der Gewinn oder Verlust des Spiels sein. Das bedeutet für die Zufallsvariable:
$$
    X(\omega) : \begin{cases}
        \Omega^2 \to \mathbb{R}\\
        \omega \mapsto \begin{cases}
            6 & \omega \in A \land \omega \notin B \\
            -6 & \omega \in A \land \omega \in B \\
            -3 & \omega \notin A \land \omega \notin B \\
            3 & \omega \notin A \land \omega \in B \\
        \end{cases}
    \end{cases}
$$

% Nicht sinnvoll mit Indikatorfunktion darzustellen, da gerade Augenzahlen zu einer ungeraden Augensumme führt und wir dadurch *nicht* die folgende Vereinfachung machen können:
% $$
%     X(\omega) : \begin{cases}
%         \Omega^2 \to \mathbb{R}\\
%         \omega \mapsto
%             \left(\mathbb{1}_A(\omega)\cdot 6 +
%             \mathbb{1}_{\overline{A}}(\omega)\cdot 3\right) \cdot \left(
%             \mathbb{1}_B(\omega) + \mathbb{1}_{\overline{B}}(\omega)\cdot (-1) \right)
%     \end{cases}
% $$

\subsubsection*{Lösung 2a}
Zur Berechnung der Wahrscheinlichkeiten betrachten wir die folgenden Zerlegungen der Grundmenge durch die Werte -6, -3, 3 und 6 für die Zufallsvariable $X$:
$$\begin{aligned}
    \set*{X {=} 6}  &= \set*{(2,2), (4,4), (6,6)} \\
    \set*{X {=} -6} &= \set*{(1,1), (3,3), (5,5)} \\
    \set*{X {=} -3} &= A\setminus B \\
                  &= \set*{
        (1,2), (1,4), (1,6), \dots, (6,1), (6,3), (6,5)
    } \\
    \set*{X {=} 3} &= B\setminus A\\
                 &= \left\{
                    (1,3), (1,5), (2,4), (2,6), (3,1), (3,5), \right.\\
                 &\quad \left. (4,2), (4,6), (5,1), (5,3), (6,2), (6,4) \right\}
\end{aligned}$$

Die Mächtigkeit der Menge $\abs*{\set*{X {=} 6}} = \abs*{\set*{X {=} -6}} = 3$, die von $\abs*{\set*{X {=} -3}} = 18$ und die von $\abs*{\set*{X {=} 3}} = 12$.
Die Mächtigkeit der Grundmenge ist $\abs*{\Omega^2} = 36$, sodass sich die Wahrscheinlichkeiten wie folgt ergeben:
$$\begin{aligned}
    P(X {=} 6)  &= \frac{3}{36} = \frac{1}{12} \\[2mm]
    P(X {=} -6) &= \frac{3}{36} = \frac{1}{12} \\[2mm]
    P(X {=} -3) &= \frac{18}{36} = \frac{1}{2} \\[2mm]
    P(X {=} 3)  &= \frac{12}{36} = \frac{1}{3} \\
\end{aligned}$$

\subsubsection*{Lösung 2b}
Der Erwartungswert $E(X)$ ist die gewichtete Summe der Zufallsvariablen auf $\Omega$ aller Ereignisse.
$$
    E(X) = \sum_{\omega\in\Omega^2} X(\omega) \cdot P(\omega)
$$

Das bedeutet hier:
$$\begin{aligned}
    E(X) &= 6\cdot P(X{=}6) + (-6)\cdot P(X{=}-6) + (-3)\cdot P(X{=}-3) + 3\cdot P(X{=}3) \\
    &= \frac{6\cdot 3}{36} - \frac{6\cdot 3}{36} - \frac{3\cdot 18}{36} + \frac{3\cdot 12}{36} \\[2mm]
    &= \frac{3\cdot 12 - 3\cdot 18}{36}\\[2mm]
    &= - \frac{18}{36} = -0,5
\end{aligned}$$


\end{document}
