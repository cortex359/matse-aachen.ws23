\documentclass[main.tex]{subfiles}

\begin{document}

\section{Aufgabe 4}
An einem Schwimmwettbewerb nehmen 20~Schwimmer:innen teil. Darunter sind 12~Auszubildende und 8~Schüler:innen.
\begin{enumerate}
    \item Wie wahrscheinlich ist es, dass die ersten drei Plätze nur von Auszubildenden eingenommen werden?
    \item Die Zufallsvariable $X$ sei nun die \glqq Anzahl der Auszubildenden unter den ersten drei Plätzen\grqq. Welches der folgenden Ereignisse ist am Wahrscheinlichsten:
    \begin{enumerate}
        \item 1 Auszubildende:r unter den ersten drei
        \item 2 Auszubildende unter den ersten drei
        \item 3 Auszubildende unter den ersten drei
    \end{enumerate}
\end{enumerate}

\subsection{Lösung 4}
%Hierbei handelt es sich strenggenommen nicht um ein Laplace Experiment, da  keine gleichmäßige, zufällige Verteilung der Ereignisse angenommen werden kann. Gehen wir jedoch davon aus, dass die Sieger:innen des Wettbewerbs nicht durch Talent, sondern durch Zufall bestimmt werden, so können wir die Wahrscheinlichkeiten wie folgt berechnen.

Der Ergebnisraum $\Omega = \set*{(\omega_1, \ldots, \omega_{20})\middle| \omega_i \in \set*{A, S} }$ hat die Mächtigkeit $\abs*{\Omega} = 2^{20} = 1.048.576$.

\subsubsection{Lösung 4a}
Die Wahrscheinlichkeit, dass unter den ersten drei Plätzen nur Auszubildende sind $A=\set*{(\omega_1, \ldots, \omega_{20}) \in \Omega \middle|\omega_{1,2,3} = A}$, beträgt
\begin{equation*}
    P(A) = \frac{12}{20} \cdot \frac{11}{19} \cdot \frac{10}{18} = \frac{11}{57} \approx 19,298\%.
\end{equation*}

\subsubsection{Lösung 4b}
Wir betrachten eine hypergeometrische Verteilung mit der Grundgesamtheit des Umfangs $N=20$, von denen $M=12$ die gewünschte Eigenschaft besitzen, beim auswählen von $n=3$ Wettbewerber:innen (Stichprobe des Umfangs $n$) genau $x$ der ersten drei Plätze zu belegen.

$$
    P(X{=}x) = \frac{\binom{M}{x}\binom{N-M}{n-x}}{\binom{N}{n}}\qquad x\in[0,n]
$$
Also gilt für das Ereignis ein Auszubildende:r unter den ersten drei
$$\begin{aligned}
    P(X{=}1) &= \frac{\binom{12}{1}\binom{20-12}{3-1}}{\binom{20}{3}} \\[3mm]
             &= \frac{\frac{12!8!}{11!6!2!}}{\frac{20!}{17!3!}} \\[3mm]
             &= \frac{3!8!12!17!}{2!6!11!20!} \\[3mm]
             &= \frac{28}{95} \approx 29,47\% \\
\end{aligned}$$
zwei Auszubildende unter den ersten drei
$$\begin{aligned}
    P(X{=}2) &= \frac{\binom{12}{2}\binom{20-12}{3-2}}{\binom{20}{3}} \\[2mm]
             &= \frac{\binom{12}{2}\cdot 8}{\binom{20}{3}} \\[2mm]
             &= \frac{44}{95} \approx 46,32\% \\
\end{aligned}$$
und für das Ereignis drei Auszubildende unter den ersten drei eine Wahrscheinlichkeit von
$$\begin{aligned}
    P(X{=}3) &= \frac{\binom{12}{3}\binom{20-12}{3-3}}{\binom{20}{3}} \\[2mm]
             &= \frac{11}{54} \approx 19,30\%. \\
\end{aligned}$$

Damit ist das Ereignis, dass zwei Auszubildende unter den ersten drei Plätzen landen $\set*{X{=}2}$ am wahrscheinlichsten. 

\end{document}
