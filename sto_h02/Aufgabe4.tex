\documentclass[main.tex]{subfiles}

\begin{document}

\section{Aufgabe 4}
An einem Schwimmwettbewerb nehmen 20~Schwimmer:innen teil. Darunter sind 12~Auszubildende und 8~Schüler:innen.
\begin{enumerate}
    \item Wie wahrscheinlich ist es, dass die ersten drei Plätze nur von Auszubildenden eingenommen werden?
    \item Die Zufallsvariable $X$ sei nun die \glqq Anzahl der Auszubildenden unter den ersten drei Plätzen\grqq. Welches der folgenden Ereignisse ist am Wahrscheinlichsten:
    \begin{enumerate}
        \item 1 Auszubildende:r unter den ersten drei
        \item 2 Auszubildende unter den ersten drei
        \item 3 Auszubildende unter den ersten drei
    \end{enumerate}
\end{enumerate}

\subsection{Lösung 4}
Hierbei handelt es sich strenggenommen nicht um ein Laplace Experiment, da  keine gleichmäßige, zufällige Verteilung der Ereignisse angenommen werden kann. Gehen wir jedoch davon aus, dass die Sieger:innen des Wettbewerbs nicht durch Talent, sondern durch Zufall bestimmt werden, so können wir die Wahrscheinlichkeiten wie folgt berechnen.

Der Ergebnisraum $\Omega = \set*{(\omega_1, \ldots, \omega_{20})\middle| \omega_i \in \set*{A, S} }$ hat die Mächtigkeit $\abs*{\Omega} = 2^{20} = 1.048.576$.

\subsubsection{Lösung 4a}
Die Wahrscheinlichkeit, dass unter den ersten drei Plätzen nur Auszubildende sind $A=\set*{(\omega_1, \ldots, \omega_{20}) \in \Omega \middle|\omega_{1,2,3} = A}$, beträgt
\begin{equation*}
    P(A) = \frac{12}{20} \cdot \frac{11}{19} \cdot \frac{10}{18} = \frac{11}{57} \approx 19,298\%.
\end{equation*}

\subsubsection{Lösung 4b}
Unter der Annahme, dass „\textbf{genau} $n$ Auszubildende unter den ersten drei“ für $n\in\set*{1,2,3}$ gemeint ist, also $P(X=n)$ erhalten wir:

$$\begin{aligned}
    P(X{=}1) &= \frac{12}{20} \cdot \frac{8}{19} \cdot \frac{7}{18} \\
    &= \frac{28}{285} \approx 9,82\% \\
    P(X{=}2) &= \frac{12}{20} \cdot \frac{11}{19} \cdot \frac{8}{18} \\
    &= \frac{44}{285} \approx 15,44\% \\
    P(X{=}3) &= \frac{12}{20} \cdot \frac{11}{19} \cdot \frac{10}{18} \\
    &= \frac{55}{285} \approx 19,30\%
\end{aligned}$$
Somit ist das Ereignis $\set*{X=3}$, also dass 3 Auszubildende unter der ersten drei sind, am wahrscheinlichsten.\\

Nimmt man jedoch an, dass nicht „genau“, sondern \textbf{mindestens} gemeint ist, also $P(X{\geq}n)$, so ergibt sich das folgende Bild:
$$\begin{aligned}
    P(X{\geq}1) &= P(X{=}1) + P(X{=}2) + P(X{=}3) \\[1mm]
    &= \frac{28 + 44 + 55}{285} \\[1mm]
    &= \frac{127}{285} \approx 44,56\% \\[4mm]
    P(X{\geq}2) &= P(X{=}2) + P(X{=}3) \\[1mm]
    &= \frac{44 + 55}{285} \\[1mm]
    &= \frac{99}{285} \approx 34,74\% \\[4mm]
    P(X{\geq}3) &= P(X{=}3) \\[1mm]
    &= \frac{11}{57} \approx 19,30\%
\end{aligned}$$
Danach wäre das Ereignis $\set*{X{\geq}1}$ am wahrscheinlichsten.


\end{document}
