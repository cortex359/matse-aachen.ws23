\documentclass[main.tex]{subfiles}

\begin{document}

\section{Aufgabe 4}
An einem Schwimmwettbewerb nehmen 20~Schwimmer:innen teil. Darunter sind 12~Auszubildende und 8~Schüler:innen.
\begin{enumerate}
    \item Wie wahrscheinlich ist es, dass die ersten drei Plätze nur von Auszubildenden eingenommen werden?
    \item Die Zufallsvariable $X$ sei nun die \glqq Anzahl der Auszubildenden unter den ersten drei Plätzen\grqq. Welches der folgenden Ereignisse ist am Wahrscheinlichsten:
    \begin{enumerate}
        \item  1 Auszubildende:r unter den ersten drei
        \item 2 Auszubildende unter den ersten drei
        \item 3 Auszubildende unter den ersten drei
    \end{enumerate}
\end{enumerate}

\subsection{Lösung 4}
Hierbei handelt es sich strenggenommen nicht um ein Laplace Experiment, da  keine gleichmäßige, zufällige Verteilung der Ereignisse angenommen werden kann. Gehen wir jedoch davon aus, dass die Sieger:innen des Wettbewerbs bereits im Vorfeld durch Zufall entschieden werden, so können wir die Wahrscheinlichkeiten wie folgt berechnen:

\begin{equation*}
    P_a = \frac{12}{20} \cdot \frac{11}{19} \cdot \frac{10}{18} 
\end{equation*}

\end{document}
