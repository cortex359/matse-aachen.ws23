\documentclass[main.tex]{subfiles}

\begin{document}

\section{Aufgabe 1}
Ein Einzelhändler hat drei DVD-Player einer bestimmten Marke geliefert bekommen und überprüft deren Funktionalität, bevor er sie an seine Kunden weitergibt. Es bezeichne nun $A_i (i = 1, 2, 3)$ das Ereignis, dass beim $i$-ten DVD-Player ein Defekt festgestellt wird. Beschreiben Sie mit Hilfe von $A_1$, $A_2$, $A_3$ und den passenden Mengenoperationen die folgenden Ereignisse.
\begin{enumerate}
    \item alle DVD-Player sind defekt,
    \item mindestens ein DVD-Player ist defekt,
    \item höchstens ein DVD-Player ist defekt,
    \item alle DVD-Player sind intakt,
    \item der erste DVD-Player ist defekt und von den beiden anderen Geräten hat höchstens eines einen Fehler,
    \item genau zwei DVD-Player sind defekt.
\end{enumerate}

\subsection{Lösung 1}
\begin{enumerate}
    \item alle DVD-Player sind defekt $$
        A_1 \cap A_2 \cap A_3
    $$
    \item mindestens ein DVD-Player ist defekt $$
        A_1 \cup A_2 \cup A_3
    $$
    \item höchstens ein DVD-Player ist defekt $$
        \complement ((A_1 \cap A_2) \cup (A_1 \cap A_3) \cup (A_2 \cap A_3))
    $$
    \item alle DVD-Player sind intakt $$
        \complement (A_1 \cup A_2 \cup A_3)
    $$
    \item der erste DVD-Player ist defekt und von den beiden anderen Geräten hat höchstens eines einen Fehler $$
        A_1 \setminus (A_1\cap A_2 \cap A_3)
    $$
    \item genau zwei DVD-Player sind defekt $$
        \left(
        (A_1 \cap A_2) \cup (A_2 \cap A_3) \cup (A_1 \cap A_3)
        \right) \setminus (A_1 \cap A_2 \cap A_3)
    $$
\end{enumerate}



\end{document}
