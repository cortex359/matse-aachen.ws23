\documentclass[main.tex]{subfiles}

\begin{document}

\section{Aufgabe 2}
Eine Urne enthält 4 rote Kugeln mit einem Kreuz, 5 rote Kugeln ohne Kreuz, 3 blaue Kugeln mit einem Kreuz, 2 blaue Kugeln ohne Kreuz, 3 weiße Kugeln mit einem Kreuz und 3 weiße Kugeln ohne Kreuz.
Wie groß ist die Wahrscheinlichkeit, beim einmaligen Ziehen einer Kugel
\begin{enumerate}
    \item eine weiße Kugel zu ziehen,
    \item eine Kugel mit Kreuz zu ziehen,
    \item eine blaue Kugel mit einem Kreuz oder eine weiße Kugel ohne Kreuz zu ziehen,
    \item eine rote Kugel oder eine Kugel mit einem Kreuz zu ziehen,
\end{enumerate}

\subsection{Lösung 2}
Es sind $n = 20$ Kugeln aus der Grundmenge $\Omega_1 = \set*{r_k, r, b_k, b, w_k, w}$ in einer Urne und es wird einmal gezogen, sodass die Wahrscheinlichkeit
\begin{enumerate}
    \item eine weiße Kugel zu ziehen $$
        P(A_{w_k} \cup A_{w}) = \frac{3+3}{10} = \frac{3}{10}
    $$
    \item eine Kugel mit Kreuz zu ziehen $$
        P(A_{r_k} \cup A_{b_k} \cup A_{w_k}) = \frac{4+3+3}{20} = \frac{1}{2}
    $$
    \item eine blaue Kugel mit einem Kreuz oder eine weiße Kugel ohne Kreuz zu ziehen $$
        P(A_{b_k} \cup A_{w}) = \frac{3+3}{20} = \frac{3}{10}
    $$
    \item eine rote Kugel oder eine Kugel mit einem Kreuz zu ziehen $$
        P(A_{r} \cup A_{r_k} \cup A_{b_k} \cup A_{w_k}) = \frac{5 + 4 + 3 + 3}{20} = \frac{3}{5}
    $$
\end{enumerate}
beträgt.

\end{document}
