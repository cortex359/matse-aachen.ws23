\documentclass[main.tex]{subfiles}

\begin{document}

\section{Aufgabe 2}
Gegeben sei folgende zweidimensionale Dichtefunktion der Zufallsvariablen $X$ und $Y$:
$$ f_{X,Y}(x,y)
= \begin{cases}
	2x + \frac{2}{3}y & \mbox{für } -0,5 \leq x \leq 0,5 \mbox{ und } 1 \leq y \leq 2 \\
	0				  & \mbox{sonst}
\end{cases} $$

\begin{enumerate}
\item Berechnen Sie die Randverteilung ($f_X(x)$ und $f_Y(y)$) für beide Zufallsvariablen $X$ und $Y$.
\item Sind die Zufallsvariablen $X$ und $Y$ stochastisch unabhängig?
\item Berechnen Sie aus der Randverteilung
\begin{enumerate}
\item den Erwartungswert $E(Y)$ sowie
\item die Varianz $\Var(Y)$.
\end{enumerate}
\end{enumerate}

\subsection{Lösung 2}

\textbf{Berechnung der Randverteilungen} \(f_X(x)\) und \(f_Y(y)\):\\
Die Randdichtefunktion \(f_X(x)\) erhält man durch Integration der gemeinsamen Dichtefunktion \(f_{X,Y}(x,y)\) über den Bereich von \(y\):
$$\begin{aligned}
	f_X(x) &= \int_{1}^{2} f_{X,Y}(x,y) \dx{y} \\[2mm]
	&= \int_{1}^{2} 2x + \frac{2}{3}y \dx{y} \\[2mm]
	&= \left[
		2xy + \frac{1}{3}y^2
	\right]_{1}^{2} \\[2mm]
	&= 4x + \frac{4}{3} - 2x - \frac{1}{3} \\[2mm]
	&= 2x + 1
\end{aligned}
$$

Die Randdichtefunktion \(f_Y(y)\) entsprechend über den Integrationsbereich von \(x\):
$$\begin{aligned}
	f_Y(y) &= \int_{-0,5}^{0,5} f_{X,Y}(x,y) \dx{x} \\[2mm]
	&= \int_{-0,5}^{0,5} 2x + \frac{2}{3}y \dx{x} \\[2mm]
	&= \left[ x^2 + \frac{2}{3}yx \right]_{-0,5}^{0,5} \\[2mm]
	&= \frac{2}{3}y
\end{aligned}$$

Wir erhalten somit $$
f_X(x) = \begin{cases}
	2x + 1 & x\in \interval{-0,5; 0,5} \\
	0 & \text{sonst}
\end{cases}$$
und $$
f_Y(y) = \begin{cases}
	\frac{2}{3}y & y\in \interval{1, 2} \\
	0 & \text{sonst}
\end{cases}
$$

\textbf{Stochastische Unabhängigkeit}:\\

Zwei Zufallsvariablen \(X\) und \(Y\) sind stochastisch unabhängig, wenn für alle \(x\) und \(y\) gilt, dass \(f_{X,Y}(x,y) = f_X(x) \cdot f_Y(y)\).
Wir überprüfen also, ob diese Bedingung erfüllt ist und wählen $x=\frac{1}{2}, y=1$ als Gegenbeispiel:
$$
	f_{X,Y}\left(\frac{1}{2}, 1\right) = 1 + \frac{2}{3} = \frac{5}{3} \neq \frac{4}{3} = 2\cdot \frac{2}{3} = f_X\left(\frac{1}{2}\right) \cdot f_Y(1)
$$
Die die Zufallsvariablen $X$ und $Y$ sind somit nicht stochastisch unabhängig.\\


\textbf{Erwartungswert und Varianz von \(Y\)}\\
Der Erwartungswert \(E(Y)\) ist das Integral von \(y \cdot f_Y(y)\) über den Bereich von \(y\).
$$\begin{aligned}
	E(Y) &= \int_{1}^{2} y\cdot f_Y(y) \dx{y} \\[2mm]
	&=\int_{1}^{2} \frac{2}{3}\cdot y^2 \dx{y} \\[2mm]
	&=\left[ \frac{2}{9} y^3 \right]_{1}^{2} \\[2mm]
	&= \frac{14}{9} \\
\end{aligned}$$

Die Varianz \(\Var(Y)\) ist das Integral von \((y - E(Y))^2 \cdot f_Y(y)\) über den Bereich von \(y\).
$$\begin{aligned}
	\Var(Y) &= \int_{1}^{2} y^2\cdot f_Y(y) \dx{y} - \left(\frac{14}{9}\right)^2\\[2mm]
	&= \int_{1}^{2} \frac{2}{3}\cdot y^3 \dx{y} - \frac{196}{81}\\[2mm]
	&= \left[ \frac{1}{6}\cdot y^4 \right]_{1}^{2} - \frac{196}{81}\\[2mm]
	&= \frac{15}{6} - \frac{196}{81}\\[2mm]
	&= \frac{13}{162}
\end{aligned}$$

\end{document}
