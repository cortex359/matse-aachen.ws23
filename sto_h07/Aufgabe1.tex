\documentclass[main.tex]{subfiles}

\begin{document}

\section{Aufgabe 1}
Die störungsfreie Laufzeit $X$ (in Stunden) eines Computers in einem Betrieb besitze folgende Dichtefunktion:
\begin{center}
$f(x)=\begin{cases}
		a \cdot \e^{-a \cdot x}	& \text{für } x \geq 0 \text{ und } a = 0,02\\
		0												& \text{für } x < 0
\end{cases}$
\end{center}
\begin{enumerate}
\item Bestimmen Sie die Verteilungsfunktion.
\item Berechnen Sie für $X$
\begin{enumerate}
\item den Erwartungswert
\item die Varianz
\item den Median
\end{enumerate}
\item Wie groß ist die Wahrscheinlichkeit, dass eine Maschine
\begin{enumerate}
\item höchstens $30$ Stunden ohne Störung läuft?
\item mindestens $40$, aber höchstens $80$ Stunden ohne Störung läuft?
\item mindestens $40$ Stunden läuft, wenn sie bereits $20$ Stunden gelaufen ist?
\end{enumerate}
\end{enumerate}

\subsection{Lösung 1a}

Die Verteilungsfunktion \(F(x)\) einer Exponentialverteilung ist das Integral der Dichtefunktion in den Grenzen von $\interval{-\infty, x}$.
$$\begin{aligned}
	\int_{-\infty}^{x} f(t) \dx{t} &= \overbrace{\int_{-\infty}^{0} f(t) \dx{t}}^{=0} + \int_{0}^{x} f(t) \dx{t} \\
	&= \int_{0}^{x} a\cdot \e^{-a\cdot t} \dx{t} \\
	&= \left[ -e^{-a\cdot t} \right]_{0}^{x} \\
	&= -e^{-a\cdot x} -\e^0 \\
	&= -e^{-a\cdot x} + 1 \\
\end{aligned}$$

Wir erhalten somit die Verteilungsfunktion $X\sim \text{Exp}(\lambda = a)$:
$$
	F(x) = \begin{cases}
		1 - \e^{-a\cdot x} & \text{für}\quad x\geq 0 \land a=0,02 \\
		0 & \text{für}\quad x < 0
	\end{cases}
$$

\subsection{Lösung 1b}

Der Erwartungswert der Zufallsvariablen $X$ ist
$$\begin{aligned}
	E(X) &= \int_{-\infty}^{\infty} x \cdot f(x) \dx{x} \\
	&= \int_{0}^{\infty} x \cdot 0,02 \e^{-0,02x} \dx{x} \\
	&= 50\ \text{Stunden.}
\end{aligned}$$

Um die Varianz zu bestimmen, bestimmen wir zunächst das zweite Moment von $X$, also $E(X^2)$, wobei $X^2$ als neue Zufallsvariable im Sinne von $X^2=g(X)$ verstanden werden kann.
$$\begin{aligned}
	E(X^2) &= \int_{-\infty}^{\infty} x^2 \cdot f(x) \dx{x} \\
	&= \int_{0}^{\infty} x^2 \cdot 0,02 \e^{-0,02x} \dx{x} \\
	&= 5.000
\end{aligned}$$

Die Varianz ist nun $\Var(X) = E(X^2)-E(X)^2 = 5.000 - 2.500 = 2.500$ Stunden$^2$.\\

Um nun den Median zu bestimmen, suchen wir ein $\mu$ sodass gilt:
$$
	P(X\leq \mu - x) = P(X\geq \mu +x) \quad \forall x\in\mathbb{R}
$$

Der Median wird auch $Q_{0.5}$ oder 0,5-Quantil genannt und halbiert die Fläche unter der Dichtefunktion. Hier ist der Median $\frac{\ln(2)}{0,02}\approx 34,6574$ Stunden.\\

Allgemein können wir bei einer Exponentailverteilung der Form $a\cdot \e^{-a\cdot x}$ also Erwartungswert, Varianz und Median direkt über die folgenden Formeln berechnen:
\begin{itemize}
\item Der Erwartungswert einer Exponentialverteilung ist $\frac{1}{a}$.
\item Die Varianz einer Exponentialverteilung ist $\frac{1}{a^2}$.
\item Der Median einer Exponentialverteilung ist $\frac{\ln(2)}{a}$.
\end{itemize}

\subsection{Lösung 1c}
Somit lassen sich die obigen Fragen wie folgt beantworten:

Die Wahrscheinlichkeit, dass ein Computer höchstens 30 Stunden ohne Störung läuft:
$$\begin{aligned}
	P(X\leq 30) &= F(30) \\
	&= 1 - \e^{-0,02 \cdot 30} \\
	&= 1 - \e^{-0,6} \\
	&\approx 0,451188 = 45,1188 \%
\end{aligned}$$

Die Wahrscheinlichkeit, dass ein Computer mindestens 40, aber höchstens 80 Stunden läuft:
$$\begin{aligned}
	P(40\leq X\leq 80) &= F(80) - F(40)\\
	&= 1 - \e^{-0,02 \cdot 80} - (1-\e^{-0,02 \cdot 40}) \\
	&= \e^{-0,8} - \e^{-1,6} \\
	&\approx 0,247432 = 24,7432 \%
\end{aligned}$$

Die Wahrscheinlichkeit, dass ein Computer mindestens 40 Stunden läuft, wenn er bereits 20 Stunden gelaufen ist:

$$\begin{aligned}
	P(X\geq 40 | X \geq 20) &= \frac{P(\set{X\geq 40}\cap \set{X\geq 20})}{P(X\geq 20)} \\[2mm]
	&= \frac{P(X\geq 40)}{P(X\geq 20)} \\[2mm]
	&= \frac{1-P(X<40)}{1-P(X<20)} \\[2mm]
	&= \frac{1-F(40)}{1-F(20)} \\[2mm]
	&\approx \frac{0,449329}{0,670320} \\[2mm]
	&\approx 0,6703 = 67,03 \% \\[2mm]
\end{aligned}$$

\end{document}
