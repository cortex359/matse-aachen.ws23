\documentclass[main.tex]{subfiles}

\begin{document}

\section{Aufgabe 1}
In Aufgabe A6 auf Aufgabenblatt Nr. 1 wurde bereits eine mobile App für die Organisation von Mitfahrgelegenheiten zur FH Aachen besprochen, die Sie auf den Markt bringen wollen. Beschreiben Sie, welche Funktionalitäten ein Minimum Viable Product (MVP) für diese Anwendung enthalten könnte. Geben Sie außerdem zwei Ausbaustufen für die Zukunft an, wenn sich der MVP auf dem Markt bewährt hat.

\subsection{Lösung 1}
Bei einem \textbf{Minimum Viable Product (MVP)} müssen die Grundfunktionalitäten so erfüllt werden, dass die Software benutzbar ist und zur Weiternutzung anregt.

Im Falle einer Mitfahrgelegenheiten-App bedeutet dies, dass Nutzende sich über Fahrten informieren können und einer Fahrt zusagen können. Wichtig ist für das MVM auch schon ein Authentifizierungssystem, da die Datensicherheit zu jeder Nutzungszeit gewährleistet sein muss. - Kurz gesagt, müssen zumindest Mitfahrgelegenheiten zustande kommen können.\\

In der \textbf{ersten Ausbaustufe} könnte die App dann um eine Option zur Bewertung der Fahrt erweitert werden oder es könnte die Möglichkeit geboten werden, mit verschiedenen Filterkriterien nach Fahrten zu suchen.\\

In einer \textbf{zweiten Ausbaustufe} könnte eine Bezahlfunktion eingerichtet werden oder etwa ein automatisiertes Matchmaking von Personen mit ähnlichem Reiseziel implementiert werden. Auch Benachrichtigungen, wenn Fahrten angeboten werden, nach denen man zuvor gesucht hat, könnte Teil einer zweiten Ausbaustufe sein.

\end{document}
