\documentclass[main.tex]{subfiles}

\begin{document}

\section{A4 Git: Änderungen rückgängig machen}
Recherchieren Sie, wie Sie die folgenden drei Herausforderungen lösen können und geben Sie
eine Lösung an:

\begin{enumerate}
\item Sie haben bei Ihrem letzten Commit ins lokale Repository leider eine Datei vergessen und Sie wollen den letzten Commit um diese Datei erweitern, ohne dass aber ein ganz neuer Eintrag im Commit-Verlauf entsteht.
\item Sie möchten die bereits ins lokale Repository per Commit übertragenen Änderungen wieder zurücknehmen, so dass dieser Commit ganz aus dem Commit-Verlauf gelöscht wird.
\item Sie möchten den letzten Commit ins Remote-Repository zurücknehmen, so dass aber die Rücknahme selbst Teil des Commit-Verlaufs wird und für andere Entwickler nachvollziehbar bleibt.
\end{enumerate}

\subsection{Lösung 4}
\begin{enumerate}
    \item Für diesen Fall gibt es \lstinline|git commit --amend|.
    \item Wenn die Commits noch nicht übertragen wurden, kann dazu \lstinline|git reset --hard <commit-hash>| verwendet werden. Sollten die Commits bereits übertragen worden sein oder möchte man keinen harten Reset durchführen, so lässt sich mit \lstinline|git revert <commit-hash>| auch die Änderungen eines Commits rückgänging machen und dies als neuen Commit speichern.
    \item \lstinline|git revert HEAD~1|
\end{enumerate}


\end{document}
