\documentclass[main.tex]{subfiles}

\begin{document}

\section{A3 Scrum}
\begin{enumerate}
    \item Erläutern Sie, wie man Ihrer Meinung nach bei Scrum vorgehen sollte, wenn das Entwicklungsteam im Laufe eines Sprints folgendes erkennt:
    \begin{enumerate}
        \item Das Sprintziel und die User Stories im Sprint Backlog können vorzeitig vor Ende des Sprints erfüllt werden.
        \item Das Sprintziel und die User Stories im Sprint Backlog können sehr wahrscheinlich nicht rechtzeitig zum Ende des Sprints erfüllt werden.
    \end{enumerate}
    \item Nennen Sie die Vor- und Nachteile einer sehr kurzen Sprintlänge (z.B. eine Woche) gegenüber einer sehr langen Sprintlänge (z.B. acht Wochen).
\end{enumerate}

\subsection{Lösung 3a}

\begin{itemize}
    \item Vorzeitige Erfüllung im Sprint
    \begin{itemize}
        \item Kommunizieren mit dem Product Owner, um zusätzliche Aufgaben zu übernehmen.
        \item Verfeinerung von begonnenen Arbeiten und Qualitätssicherung.
        \item Sicherstellen, dass das Sprintziel und die Qualität erhalten bleiben.
    \end{itemize}
    \item Nicht rechtzeitige Erfüllung im Sprint
    \begin{itemize}
        \item Sofortige Kommunikation mit dem Product Owner.
        \item Priorisierung der verbleibenden Aufgaben.
        \item Möglicherweise das Abschneiden oder Verschieben von Aufgaben.             
    \end{itemize}
\end{itemize}

\subsection{Lösung 3a}
\begin{itemize}
    \item Kurze Sprints (eine Woche)
    \begin{itemize}
        \item Vorteile: Hohe Flexibilität, häufiges Feedback, schnellere Lieferung von Features, bessere Fehlererkennung.
        \item Nachteile: Höherer Verwaltungsaufwand, Druck auf das Team, schwieriger für remote Teams.
    \end{itemize}
    \item Lange Sprints (acht Wochen)
    \begin{itemize}
        \item Vorteile: Weniger Verwaltungsaufwand, mehr Zeit für tiefgreifende Arbeit, geringerer Zeitdruck.
        \item Nachteile: Geringere Flexibilität, längere Lieferzyklen, schwierigere Fehlererkennung.
    \end{itemize}
\end{itemize}

Die Wahl der Sprintlänge hängt von den individuellen Projektanforderungen und der Balance zwischen Flexibilität und Gründlichkeit ab.

\end{document}
