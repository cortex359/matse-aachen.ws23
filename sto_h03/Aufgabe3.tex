\documentclass[main.tex]{subfiles}

\begin{document}

\section{Aufgabe 3}
Auf einem Tisch stehen $n$ äußerlich nicht unterscheidbare Urnen $U_1, U_2, \ldots, U_n$ wobei in Urne $U_i$ genau $i$ weiße und $n - i$ andersfarbige Kugeln sind. Aus einer zufällig ausgewählten Urne werde eine Kugel gezogen.

Wie groß ist die Wahrscheinlichkeit, dass die Kugel weiß ist?

\subsection{Lösung 3}
Die Wahrscheinlichkeit aus Urne $U_i$ zu ziehen ist $P(U_i) = \frac{1}{n}$. Die Wahrscheinlichkeit, dass die dort gezogene Kugel weiß ist, ist $P(W|U_i) = \frac{i}{n}$. Gesucht ist $P(W)$, welche wir mit der Formel der totalen Wahrscheinlichkeit wie folgt ermitteln können:
$$\begin{aligned}
    P(W) &= \sum^{n}_{i=1} P(U_i)\cdot P(W|U_i) \\[2mm]
    &= \sum^{n}_{i=1} \frac{1}{n} \cdot \frac{i}{n} \\[2mm]
    &= \frac{1}{n^2} \cdot \sum^{n}_{i=1} i \\[2mm]
    &= \frac{1}{n^2} \cdot \frac{(n+1)\cdot n}{2} \\[2mm]
    &= \frac{n+1}{2n} \\
\end{aligned}$$

\end{document}
