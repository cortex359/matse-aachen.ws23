\documentclass[main.tex]{subfiles}

\begin{document}

\section{Aufgabe 5}
Zwei Abwasserpumpen arbeiten völlig unabhängig voneinander (Redundanz). Nach Auswertung der Wartungshefte zeigt sich, dass die neue Pumpe eine Ausfallwahrscheinlichkeit von 5\%, die ältere von 10\% hat.
Die Wahrscheinlichkeit für den gleichzeitigen Ausfall beider Pumpen beträgt 0,5\%. Da ein Notbetrieb mit einer Pumpe nur kurzzeitig
möglich ist, ist die Wahrscheinlichkeit für das Eintreten dieses Notbetriebes gesucht.

\subsection{Lösung 5}
Wir unterscheiden den Ausfall der neueren Pumpe $A$ und den Ausfall der älteren Pumpe $B$. Es ist gegeben, dass $P(A) = 0,05$, $P(B) = 0,1$ und $P(A\cap B) = 0,005$. 

Gesucht ist die Wahrscheinlichkeit, dass eine der beiden Pumpen, aber nicht beide Pumpen gleichzeitig ausfallen $P(A\cup B\setminus(A\cap B))$, welche sich, aufgrund der stochastischen Unabhängigkeit beider Ereignisse, wie folgt berechnen lässt:
$$\begin{aligned}
    P(A\cup B\setminus (A\cap B)) &= P(A) + P(B) - P(A\cap B) - P(A\cap B)\\
    &= 0,05 + 0,1 - 2\cdot 0,005 \\
    &= 14\%
\end{aligned}
$$

\end{document}
