\documentclass[main.tex]{subfiles}

\begin{document}

\section{Aufgabe 1}
Bei einer Geschwindigkeitskontrolle registrierte die Verkehrspolizei, dass 20\% der Fahrzeuge mit überhöhter Geschwindigkeit fuhren. 20\% der Fahrer und Fahrerinnen, die aufgrund der zu hohen Geschwindigkeit angehalten wurden, standen unter Alkoholeinfluss. Von den übrigen Kraftfahrern und -fahrerinnen, die nicht zu schnell fuhren, wurden mittels Stichproben ermittelt, dass 5\% von ihnen ebenfalls zu hohe Promillewerte Alkohol
aufwiesen.

\begin{enumerate}
    \item Wie viel Prozent aller Kraftfahrer und -fahrerinnen, die die betreffende Kontrolle passiert haben, standen unter Alkoholeinfluss?
    \item Wie viel Prozent der Kraftfahrer und -fahrerinnen, die Alkohol getrunken hatten, fuhren zu schnell?
\end{enumerate}

\subsection{Lösung 1}
\textit{Hinweis:} Wir nehmen an, dass die Polizei alle Fahrzeuge, bei denen eine überhöhte Geschwindigkeit registriert wurde, auch angehalten hat, also „Registrieren“ und „Anhalten“ gleichbedeutend sind.\\

Wir lesen $P(S) = 0,2$ ist die Wahrscheinlichkeit, dass ein Auto zu schnell war und $P(A|S) = 0,2$, dass der Fahrer dabei alkoholisiert war.
Dann ist $P(\overline{S}) = 0,8$ die Wahrscheinlichkeit, dass ein Auto nicht zu schnell war und dem Text entnehmen wir weiter, dass $P(A|\overline{S}) = 0,05$ die Wahrscheinlichkeit ist, dass der Fahrer dabei alkoholisiert war. \\

Der Anteil der Fahrenden, welche unter Alkoholeinfluss standen $P(A)$, setzt sich nun wie folgt zusammen:
$$\begin{aligned}
    P(A) &= P(S)\cdot P(A|S) + P(\overline{S})\cdot P(A|\overline{S}) \\
         &= 0,2 \cdot 0,2 + 0,8 \cdot 0,05 \\
         &= 0,08 = 8\%
\end{aligned}$$

Der Anteil der Fahrenden, welche unter Alkoholeinfluss zu schnell fuhren $P(S|A)$ ist 
$$\begin{aligned}
    P(S|A) &= \frac{P(S)\cdot P(A|S)}{P(S)\cdot P(A|S) + P(\overline{S})\cdot P(A|\overline{S})} \\[3mm]
           &= \frac{0,2 \cdot 0,2}{0,2 \cdot 0,2 + 0,8 \cdot 0,05}\\[2mm]
           &= \frac{1}{2} = 50\%.
\end{aligned}$$



\end{document}
