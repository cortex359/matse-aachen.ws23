\documentclass[main.tex]{subfiles}

\begin{document}

\section{Aufgabe 2}
Das Abwassersystem einer Gemeinde, an das $1.332$ Haushalte angeschlossen sind, ist für eine maximale Last von $13.500$ Litern pro Stunde ausgelegt. \\[1mm]
Nehmen Sie an, dass die einzelnen Abwassermengen (pro Stunde) von $n$ angeschlossenen Haushalten beschrieben werden können durch stochastisch unabhängige Zufallsvariablen $X_1$, ..., $X_n$, wobei $X_i$ für $i \in \{ 1, \cdots, n \}$ normalverteilt ist mit Erwartungswert $\mu = 10$ (Liter/ Stunde) und Varianz $\sigma^2 = 4$ ((Liter/ Stunde)$^2$). Berechnen Sie 
\begin{enumerate}
\item den Erwartungswert und die Varianz für die $1.332$ angeschlossenen Haushalte.
\item die Wahrscheinlichkeit einer Überlastung des Abwassersystems (für $1.332$ angeschlossene Haushalte).
\end{enumerate}

\subsection{Lösung 2}

\end{document}
