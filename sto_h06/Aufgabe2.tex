\documentclass[main.tex]{subfiles}

\begin{document}

\section{Aufgabe 2}
Das Abwassersystem einer Gemeinde, an das $1.332$ Haushalte angeschlossen sind, ist für eine maximale Last von $13.500$ Litern pro Stunde ausgelegt. \\[1mm]
Nehmen Sie an, dass die einzelnen Abflussraten von $n$ angeschlossenen Haushalten durch stochastisch unabhängige Zufallsvariablen $X_1$, \dots, $X_n$ beschrieben werden können, wobei $X_i$ für $i \in \{ 1, \dots, n \}$ normalverteilt ist mit Erwartungswert $\mu = 10$ [Liter/ Stunde] und Varianz $\sigma^2 = 4$ [Liter$^2$ / Stunde$^2$].\\

Berechnen Sie 
\begin{enumerate}
\item den Erwartungswert und die Varianz für die $1.332$ angeschlossenen Haushalte.
\item die Wahrscheinlichkeit einer Überlastung des Abwassersystems (für $1.332$ angeschlossene Haushalte).
\end{enumerate}

\subsection{Lösung 2}
$$
    X\sim \mathcal{N}(\mu {=} 10, \sigma^2 {=} 4)
$$

Der Erwartungswert beträgt
$$
    E\left(\bigcap_{i\in\set{1,\dots,n}} X_i\right) = n\cdot \mu = 1.332\cdot 10 = 13.320
$$
also 13.320 [Liter / Stunde] und die Varianz $$
    \Var\left(\bigcap_{i\in\set{1,\dots,n}} X_i\right) = 1.332 \cdot 4 = 5.328
$$
5.328 [Liter$^2$ / Stunde$^2$].\\
% warum geht 13.320^2 - 1332\cdot 10^2 nicht?

Um die Wahrscheinlichkeit einer Überlastung des Abwassersystems $P\left(\bigcap_{i\in\set{1,\dots,n}} X_i {>} 13.500\right)$ zu berechnen, sagen wir $S_n := \bigcap_{i\in\set{1,\dots,n}} X_i$ und betrachten die Normalverteilung mit den oben bestimmten Parametern $S_n \sim \mathcal{N}(\mu {=} 13.320, \sigma^2 {=} 5.328)$.

$$\begin{aligned}
    P(S_n {>} 13.500) &= 1 - P(S_n {\leq} 13.500) \\
    &= 1 - P\left(\frac{S_n - \mu}{\sigma} {\leq} \frac{13.500-13.320}{\sqrt{5.328}}\right) \\
    &= 1 - P\left(\frac{S_n - \mu}{\sigma} {\leq} \frac{180}{\sqrt{5.328}}\right) \\
    &\approx 1 - \Phi (2,47)\\
    &= 1 - 0,99324 = 0,676 \%
\end{aligned}$$

Die Wahrscheinlichkeit, dass es zu einer Überlastung des Abwassersystems für 1.332 angeschlossene Haushalte kommt, beträgt 0,676\%. 

\end{document}
