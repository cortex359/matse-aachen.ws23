\documentclass[main.tex]{subfiles}

\begin{document}

\section{Aufgabe 1}
Die Wahrscheinlichkeit, einen bestimmten Antikörper zu haben, sei gleich $4\%$.
\begin{enumerate}
\item Modellieren Sie die Zufallsvariable, die die Anzahl der Antikörper-Träger unter $10.000$ Untersuchten zählt (Verteilung angeben).
\item Berechnen Sie näherungsweise die Wahrscheinlichkeit, dass bei der Untersuchung von $10.000$ Personen zwischen $300$ und $500$ Personen den Antikörper haben.
\end{enumerate}

\subsection{Lösung 1}
Die Zufallsvariable $X = \set{\#\text{Antikörper-Träger unter 10k}}$ wird durch die Binominalverteilung $X\sim\Bin(n{=}10.000, p{=}0,04)$ modelliert. 

Entsprechend lautet die Wahrscheinlichkeitsfunktion
$$
    P(X{=}x) = \binom{10.000}{x} \cdot 0,04^{x} \cdot 0,96^{10.000-x}.
$$

Um nun approximativ die Wahrscheinlichkeit $P(500{>}X{>}300)$ zu berechnen, prüfen wir zunächst die Approximationsbedingungen.
Da wir ein großes $n$ bei kleinem $p$ vorliegen haben, beginnen wir zunächst mit der Approximationsbedingung für die Approximation durch eine Poisson-Verteilung:
$$
    n\cdot p = 10.000 \cdot 0,04 = 400 \not\leq 10
$$

Approximationsbedingung für Normalverteilung:
$$
    n\cdot p\cdot (1-p) = 10.000\cdot 0,04 \cdot 0,96 = 384 > 9 \quad \checkmark
$$

Die zweite Bedingung ist erfüllt, sodass wir mit einer Normalverteilung approximieren können.
$$
    X\sim\Bin(n{=}10.000, p{=}0,04) \approx \mathcal{N}(\mu, \sigma^2)
$$
Da bei einer Normalverteilung $E(X) = \mu = n\cdot p$ gilt und für $\Var(X)=np\cdot (1-p)$ ist, erhalten wir 
$$
    X\sim\Bin(n{=}10.000, p{=}0,04) \approx \mathcal{N}(\mu {=} 400, \sigma^2 {=} 384).
$$

Statt $$
    P(300 {\leq} X {\leq} 500) = \sum_{j=300}^{500} \binom{10.000}{j} \cdot 0,04^j \cdot 0,96^{10.000 - j}
$$
% 0.9999995843225466982491241222304726674856405917540672356731066173…
können wir nun über die Approximation (ohne Stetigkeitskorrektur) Folgendes berechnen:
$$\begin{aligned}
    P(300 {\leq} X {\leq} 500) &\approx 
        \Phi \left(\frac{500-400}{\sqrt{400\cdot 0,96}}\right) -
        \Phi \left(\frac{300-400}{\sqrt{400\cdot 0,96}}\right) \\
        &\approx \Phi(5,10310) - \Phi(-5,10310) \\
        &= 2\cdot \Phi(5,10310) - 1 \\
        &= 1
\end{aligned}$$

Bei der Untersuchung von 10.000 Personen zwischen 300 und 500 Personen mit Antikörpern anzutreffen ist somit ein \textbf{sicheres Ereignis}.



\end{document}
