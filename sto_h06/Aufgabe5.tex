\documentclass[main.tex]{subfiles}

\begin{document}

\section{Aufgabe 5}
Bei der Reinigung eines Kühlschrankes im Haushalt lässt sich eine leichte Verstellung des Thermostates nie ganz vermeiden. Wir fassen die sich nach der Reinigung einstellende Temperatur als eine normalverteilte Zufallsvariable mit dem Erwartungswert $3^{\circ}C$ und der Varianz $9^{\circ}C^2$ auf.
\begin{enumerate}
\item Wie groß ist die Wahrscheinlichkeit, dass eine Temperatur den kritisch angegebenen Wert von $9^{\circ}C$ übersteigt?
\item Wie wahrscheinlich ist es, dass der Gefrierpunkt von $0^{\circ}C$ unterschritten wird?
\item Welche Temperatur wird mit einer Wahrscheinlichkeit von $99\%$ nicht überschritten?
\end{enumerate}

\subsection{Lösung 5}
Sei $X = \set{\text{Temperatur}} \sim\mathcal{N}(\mu {=} 3, \sigma^2 {=} 9)$. 
\subsubsection{5a)}
Wir bestimmen $P(X{>}9)$ wie folgt:
$$\begin{aligned}
    P(X{>}9) &= 1 - P(X{\leq}9) \\
    &= 1 - P\left(\frac{X-\mu}{\sigma} \leq \frac{9-3}{\sqrt{9}}\right) \\
    &= 1 - \Phi(2) \\
    &\approx 1 - 0,97725 = 0,02275 \\
    &= 2,275 \%
\end{aligned}$$
Die Wahrscheinlichkeit, dass der Kühlschrank die kritische Temperatur von 9$^\circ$C übersteigt beträgt 2,275 \%.

\subsubsection{5b)}
$$\begin{aligned}
    P(X{\leq}0) &= P\left(\frac{X-\mu}{\sigma} \leq \frac{0-3}{\sqrt{9}}\right) \\
    &= \Phi(-1) \\
    &= 1 - \Phi(1) \\
    &\approx 0,15866 \\
    &= 15,866 \%
\end{aligned}$$
Die Wahrscheinlichkeit, dass der Gefrierpunkt unterschritten wird, beträgt 15,866 \%. 

\subsubsection{5c)}
\begin{equiveqs}
       & P(X{\leq}k) &= 0,99 \\[3mm]
\equiv & P\left(\frac{X-\mu}{\sigma} \leq \frac{k-3}{\sqrt{9}}\right) &= 0,99\\[4mm]
\equiv & \Phi \left(\frac{k-3}{3}\right) &= 0,99 \\[3mm]
\equiv & \frac{k-3}{3} &= \Phi^{-1}(0,99) \\[3mm]
\Rightarrow & \frac{k-3}{3} &\approx 2,33 \\[3mm]
\equiv & k &\approx 9,99 \\
\end{equiveqs}
Eine Temperatur von rund 10$^\circ$C wird mit einer Wahrscheinlichkeit von 99\% nicht überschritten. 

\end{document}
