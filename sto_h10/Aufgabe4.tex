\documentclass[main.tex]{subfiles}

\begin{document}

\section{Aufgabe 4}
Ein Sportlehrer hat eine Schulklasse in $2$ Mannschaften aufgeteilt. Während die 1. Mannschaft gegen die Nachbarklasse Fußball spielt, trainiert die 2. Mannschaft Kugelstoßen. Nach einem Monat werden die Leistungen im Kugelstoßen verglichen. \\[2mm]
%
Die $11$ Schülerinnen und Schüler der 1. Mannschaft erzielen im Durchschnitt eine Weite von $6,54$ m mit einer empirischen Stichprobenvarianz von $1,68$ m$^{2}$. Die $16$ Schülerinnen und Schüler der 2. Mannschaft erzielen im Durchschnitt eine Weite von $7,85$ m mit einer empirischen Stichprobenvarianz von $2,04$ m$^{2}$. \\[2mm]
%
Ist der Unterschied der Mittelwerte signifikant bei einem Signifikanzniveau von $5\%$? \\[2mm]
%
\textit{Hinweis}: Sie können die Gleichheit der unbekannten Varianz voraussetzen.

\subsection{Lösung 4}

Stichprobe 1: $$
n_1=11,\quad \mu_1 = 6,54,\quad s_1^2=1,68
$$
Stichprobe 2: $$
n_2=16,\quad \mu_2 = 7,85,\quad s_2^2=2,04
$$

Unter der Annahme, dass die Leistungen der Schlüler:innen jeweils normalverteilt sind, eignet sich der Zweistichproben-t-Test um den Vergleich der Mittelwerte durchzuführen. Mit einem Signifikanzniveau von $\alpha = 0,05$ untersuchen wir, ob die Mittelwerte sich unterscheiden, also $$
    H_0: \mu_1 = \mu_2 \qquad \text{gegen} \qquad H_1: \mu_1 \neq \mu_2.
$$

Wir bestimmen $$\begin{aligned}
    T &= \sqrt{\frac{n_1\cdot n_2}{n_1+n_2}} \cdot \frac{\mu_1 - \mu_2}{s} \\
    &= \sqrt{\frac{11\cdot 16}{11+16}} \cdot \frac{6,54 - 7,85}{s} \\
\end{aligned}$$

wobei sich das unbekannt $s$ durch $$\begin{aligned}
    % @NOTE: Wichtige Formel, nochmal überprüfen
    s &= \sqrt{\frac{1}{n_1+n_2 -2} \cdot \left((n_1 - 1) \cdot s^2_1 + (n_2 - 1) \cdot s^2_2\right)} \\[3mm]
    &= \sqrt{\frac{1}{11 + 16 -2} \cdot \left(10\cdot 1,68 + 15\cdot 2,04\right)} \\[3mm]
    &\approx 1,37695
\end{aligned}$$
ergibt. Somit erhalten wir die Teststatistik
$$\begin{aligned}
    T &= \sqrt{\frac{11\cdot 16}{11+16}} \cdot \frac{6,54 - 7,85}{1,37695} \\
    &\approx -2,42900.
\end{aligned}$$

Der kritische t-Wert für ein zweiseitiges 95\%-Konfidenzintervall mit $n_1+n_2 - 2 = 25$ Freiheitsgraden und $p=1-\frac{\alpha}{2} = 0,975$ ist etwa $t_{\text{kritisch}} = 2,060$.\\

Da der Absolutwert der Teststatistik $\abs{T} = 2,429 > t_{\text{kritisch}}$ ist, lehnen wir die Nullhypothese zu Gunsten von $H_1$ ab.\\

Dies bedeutet, dass der Unterschied in den Mittelwerten der beiden Mannschaften statistisch signifikant, mit einem Signifikanzniveau von 5\% ist.


\end{document}
