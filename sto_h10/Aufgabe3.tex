\documentclass[main.tex]{subfiles}

\begin{document}

\section{Aufgabe 3}
Ein Unternehmer vertritt die Meinung, dass ein von ihm eingeführtes Geschäftsmodell den Umsatz seines Unternehmens gesteigert hat. Vor der Einführung des Modells hat das Unternehmen einem Umsatz von durchschnittlich $2,45$ Mio. Euro pro Monat erwirtschaftet. In den acht Monaten seit Einführung des neuen Modells wurden folgende monatlichen Umsätze (in Mio. Euro) erzielt:
\begin{center}
	\begin{tabular}{|l|c|c|c|c|c|c|c|c|} \hline
		Monat  &    $1$ &    $2$ &    $3$ &    $4$ &    $5$ &    $6$ &    $7$ &    $8$ \\ \hline
		Umsatz & $2,39$ & $2,55$ & $2,51$ & $2,45$ & $2,62$ & $2,53$ & $2,41$ & $2,59$ \\ \hline	
	\end{tabular}
\end{center}
\begin{enumerate}
\item Bestimmen Sie einen geeigneten Test zur Überprüfung der Aussage des Unternehmens. Gehen Sie hierbei davon aus, dass man die Monatsumsätze als Realisationen stochastisch unabhängiger, jeweils $\mathcal{N}(\mu; \sigma^2)$-verteilter Zufallsvariablen mit $\mu \in \mathbb{R}$ und $\sigma > 0$ ansehen kann.
\item Können Sie die Aussage des Unternehmens anhand der gegebenen Daten zum Niveau $\alpha = 0,05$ bestätigen?
\end{enumerate}

\subsection{Lösung 3}

\end{document}
