\documentclass[main.tex]{subfiles}

\begin{document}

\section{Aufgabe 3}
Ein Unternehmer vertritt die Meinung, dass ein von ihm eingeführtes Geschäftsmodell den Umsatz seines Unternehmens gesteigert hat. Vor der Einführung des Modells hat das Unternehmen einem Umsatz von durchschnittlich $2,45$ Mio. Euro pro Monat erwirtschaftet. In den acht Monaten seit Einführung des neuen Modells wurden folgende monatlichen Umsätze (in Mio. Euro) erzielt:
\begin{center}
	\begin{tabular}{|l|c|c|c|c|c|c|c|c|} \hline
		Monat  &    $1$ &    $2$ &    $3$ &    $4$ &    $5$ &    $6$ &    $7$ &    $8$ \\ \hline
		Umsatz & $2,39$ & $2,55$ & $2,51$ & $2,45$ & $2,62$ & $2,53$ & $2,41$ & $2,59$ \\ \hline	
	\end{tabular}
\end{center}
\begin{enumerate}
\item Bestimmen Sie einen geeigneten Test zur Überprüfung der Aussage des Unternehmens. Gehen Sie hierbei davon aus, dass man die Monatsumsätze als Realisationen stochastisch unabhängiger, jeweils $\mathcal{N}(\mu; \sigma^2)$-verteilter Zufallsvariablen mit $\mu \in \mathbb{R}$ und $\sigma > 0$ ansehen kann.
\item Können Sie die Aussage des Unternehmens anhand der gegebenen Daten zum Niveau $\alpha = 0,05$ bestätigen?
\end{enumerate}

\subsection{Lösung 3}

Seien $X_1, \dots, X_N$ unabhängig und $X_i = \set{\text{Monatsumsatz}} \sim \mathcal{N}(\mu, \sigma^2)$, sowie $\mu_0 = 2,45$~M€ pro Monat. Bei einer Stichprobe vom Umfang $n=8$ soll nun überprüft werden, ob $H_0: \mu \leq \mu_0$ (also nach Einführung des Geschäftsmodells keine Steigerung stattgefunden hat) oder $H_1: \mu > \mu_0$ (der Umsatz gestiegen ist).\\

Ein geeigneter Test zur Überprüfung der Annahme ist der Einstichproben-t-Test mit der Nullhypothese $H_0: \mu \leq 2,45$.\\

Wir berechnen den t-Wert mit der Formel $$
	t = \sqrt{n}\cdot \frac{\overline{x}-\mu_0}{s}
$$
wobei wir den Durchschnitt $\overline{x} = 2,50625$~M€ und die Standardabweichung $s = 0,083141$ aus der Stichprobe berechnen können.\\

Wir erhalten somit einen t-Wert von $$\begin{aligned}
	t &= \sqrt{n}\cdot \frac{\overline{x}-\mu_0}{s} \\
	&= \sqrt{8} \cdot \frac{2,50625-2,45}{0,083141} \\
	&\approx 1,91360
\end{aligned}$$
welchen wir mit dem kritischen t-Wert für $\alpha=0,05$ bei $n - 1 =7$  Freiheitsgraden und erhalten aus der Tabelle $t_{\text{kritisch}} = 1,895$. 

Da der berechnete t-Wert $t = 1,914$ größer als der kritische t-Wert $t_{\text{kritisch}} = 1,895$ ist, können wir die Nullhypothese $H_0: \mu \leq 2,45$ zum Signifikanzniveau $\alpha = 0,05$ ablehnen. Dies bedeutet, dass die Daten darauf hindeuten, dass das neue Geschäftsmodell den Durchschnittsumsatz tatsächlich erhöht hat.

\end{document}
