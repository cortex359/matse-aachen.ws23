\documentclass[main.tex]{subfiles}

\begin{document}

\section{Aufgabe 2}
Das durchschnittliche Geburtsgewicht von Kindern von Nichtraucherinnen beträgt $3.500$ g. In einer Studie soll gezeigt werden, dass das Geburtsgewicht von Kindern, deren Mütter starke Raucherinnen sind, im Durchschnitt niedriger ist als $3.500$ g. Dazu wird das Geburtsgewicht von $n$ Kindern gemessen, deren Mütter während der Schwangerschaft täglich mehr als $20$ Zigaretten rauchten. Wir nehmen dabei an, dass die Geburtsgewichte als Realisierungen unabhängiger Zufallsvariablen mit gleicher Normalverteilung $\mathcal{N}(\mu; 350^2)$ modelliert werden können.
\begin{enumerate}
\item Formulieren Sie eine geeignete Hypothese und Alternative, um die obige Behauptung "`statistisch nachzuweisen"'.
\item Zu welchem Ergebnis kommt der Test bei einem Signifikanzniveau von $\alpha = 5\%$, wenn bei den $20$ Kindern ein durchschnittliches Geburtsgewicht von $\overline{x}_{20} = 3.200$ g gemessen wurde?
\item Mit welcher Wahrscheinlichkeit wird mit der in b) ermittelten Entscheidungsregel eine Fehlentscheidung getroffen, wenn im Fall $n = 20$ das durchschnittliche Geburtsgewicht in Wirklichkeit $3.400$ g beträgt?
\end{enumerate}

\subsection{Lösung 2}
Seien die Zufallsvariablen $X_1, \dots, X_n$ unabhängig und $X_i = \set{\text{Geburtsgewicht}}\sim \mathcal{N}(\mu, 350^2)$.\\

\subsubsection{Lösung 2a}
Die Studie untersucht, ob das durchschnittliche Geburtsgewicht von Kindern, deren Mütter während der Schwangerschaft mehr als 20 Zigaretten am Tag geraucht haben, kleiner als $3.500$~g ist. Als Nullhypothese können wir also $H_0: \mu \geq 3.500$~g wählen. Die Gegenhypothese dazu ist $H_1: \mu < 3.500$~g.

\subsubsection{Lösung 2b}
Da die Standardabweichung $\sigma = 350$~g bekannt und $X\sim\mathcal{N}$ ist, können wir den Gauß-Test mit einem Signifikanzniveau von $\alpha = 0,05$ anwenden. 
Bei $n=20$ Kindern und einem durchschnittlichen Geburtsgewicht von $\overline{X}_{20} = 3.200$~g. 

Wir berechnen die Teststatistik $Z$ mit
$$\begin{aligned}
	Z &= \sqrt{n} \cdot \frac{\overline{X}_{20} - \mu_0}{\sigma} \\[2mm]
    &= \sqrt{20} \cdot \frac{3.200 - 3.500}{350} \\[2mm]
    & -\frac{12\sqrt{5}}{7} \\[2mm]
    &\approx -3,833259.
\end{aligned}$$

Die Entscheidungsregel lautet: Lehne $H_0$ ab, wenn $Z$ kleiner als das (kritische) $(1-\alpha)$-Quantil der Standardnormalverteilung ist.\\

Das kritische Quantil für ein Signifikanzniveu von 5\% bei einem einseitigen Test beträgt nach Tabelle etwa $u_{0,05} = - u_{0,95} = - 1,64$.\\

Da $Z = -3,8 < -1,64 = u_{0,05}$ ist, lehnen wir die die Nullhypothese ab. Das heißt, dass hinreichend sicher ist, dass das durchschnittliche Geburtsgewicht von Kindern, deren Mütter starke Raucherinnen sind, statistisch signifikant niedriger als 3.500 g ist.

\subsubsection{Lösung 2c}
Gefragt ist nach der Wahrscheinlichkeit für einen Fehler zweiter Art (einem Typ-II-Fehler), nämlich der Wahrscheinlichkeit, eine tatsächlich falsche Nullhypothese $H_0$ nicht abzulehnen, wenn bei einer Stichprobe vom Umfang $n=20$ das durchschnittliche Geburtsgewicht $3.400$ g beträgt. 

$$\begin{aligned}
    \beta(\mu=3.400) &= 1 - g(3.400) \\
    &= 1 - P_\mu \left( \sqrt{n} \cdot \frac{\overline{X}_n - \mu_0}{s_n} < t_{n-1; \alpha} \right) \\
    &= 1 - \Phi \left(\sqrt{n}\cdot \frac{\mu_0 - \mu}{s_n} + u_{\alpha} \right) \\
    &= 1 - \Phi \left(\sqrt{20}\cdot \frac{3.500-3.400}{350} -1,64 \right) \\
    &= 1 - \Phi \left( -0,362247 \right) \\
    &= \Phi \left( 0,362247 \right)\\
    &\approx 0,64058 = 64,058\%
\end{aligned}$$
Das bedeutet, dass die Wahrscheinlichkeit, dass wir die Nullhypothese $H_0$ fälschlicherweise beibehalten, wenn das durchschnittliche Geburtsgewicht $3.400$~g beträgt, etwa 64\% ist.

\end{document}
