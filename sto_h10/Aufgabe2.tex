\documentclass[main.tex]{subfiles}

\begin{document}

\section{Aufgabe 2}
Das durchschnittliche Geburtsgewicht von Kindern von Nichtraucherinnen beträgt $3.500$ g. In einer Studie soll gezeigt werden, dass das Geburtsgewicht von Kindern, deren Mütter starke Raucherinnen sind, im Durchschnitt niedriger ist als $3.500$ g. Dazu wird das Geburtsgewicht von $n$ Kindern gemessen, deren Mütter während der Schwangerschaft täglich mehr als $20$ Zigaretten rauchten. Wir nehmen dabei an, dass die Geburtsgewichte als Realisierungen unabhängiger Zufallsvariablen mit gleicher Normalverteilung $\mathcal{N}(\mu; 350^2)$ modelliert werden können.
\begin{enumerate}
\item Formulieren Sie eine geeignete Hypothese und Alternative um die obige Behauptung "`statistisch nachzuweisen"'.
\item Zu welchem Ergebnis kommt der Test bei einem Signifikanzniveau von $\alpha = 5\%$, wenn bei den $20$ Kindern ein durchschnittliches Geburtsgewicht von $\overline{x}_{20} = 3.200$ g gemessen wurde?
\item Mit welcher Wahrscheinlichkeit wird mit der in b) ermittelten Entscheidungsregel eine Fehlentscheidung getroffen, wenn im Fall $n = 20$ das durchschnittliche Geburtsgewicht in Wirklichkeit $3.400$ g beträgt?
\end{enumerate}

\subsection{Lösung 2}



\end{document}
