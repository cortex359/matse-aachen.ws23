\documentclass[main.tex]{subfiles}

\begin{document}

\section{Aufgabe 1}
Der durchschnittliche Kraftstoffverbrauch (in Liter/100 km) von Autos eines bestimmten Typs kann durch eine $\mathcal{N}(\mu; \sigma^2)$-verteilte Zufallsvariable mit $\mu \geq 0$ und $\sigma > 0$ beschrieben werden. Der Hersteller gibt an, dass Fahrzeuge diesen Types einen Durchschnittsverbrauch von weniger als $8,5$ Liter/100 km haben. \\[2mm]
%
Entscheiden Sie, welches der folgenden Testverfahren zur Überprüfung der Aussage des Herstellers zu einem Signifikanzniveau $\alpha \in (0;1)$ geeignet ist. Begründen Sie ihre Antwort.
\begin{enumerate}
	\item[1)] der $t$-Test mit den Hypothesen $H_0: \mu \leq 8,5$ gegen $H_1: \mu > 8,5$
	\item[2)] der Gauß-Test mit den Hypothesen $H_0: \mu \geq 8,5$ gegen $H_1: \mu < 8,5$
	\item[3)] der $t$-Test mit den Hypothesen $H_0: \mu \geq 8,5$ gegen $H_1: \mu < 8,5$
	\item[4)] der $t$-Test mit den Hypothesen $H_0: \mu = 8,5$ gegen $H_1: \mu \neq 8,5$
\end{enumerate}

\subsection{Hypothesentests}

\textbf{Gauß-Test:}
\begin{itemize}
\item \textbf{Einsatzgebiet:} Wenn Sie eine Hypothese über den Mittelwert einer normalverteilten Grundgesamtheit testen möchten und die Varianz der Grundgesamtheit bekannt ist.
\item \textbf{Typische Situationen:} Dieser Test wird oft in Qualitätssicherungsprozessen oder bei der Überprüfung von Herstellungsstandards eingesetzt, wo die Varianz aus historischen Daten bekannt ist.
\item \textbf{Annahmen:} Die Stichprobenwerte müssen unabhängig und identisch verteilt sein, und die Grundgesamtheit muss normalverteilt sein.
\end{itemize}

\textbf{Einstichproben-t-Test:}
\begin{itemize}
\item \textbf{Einsatzgebiet:} Dieser Test wird angewendet, wenn Sie eine Hypothese über den Mittelwert einer normalverteilten Grundgesamtheit testen möchten, aber die Varianz der Grundgesamtheit ist nicht bekannt.
\item \textbf{Typische Situationen:} Der Test ist nützlich in der Forschung und anderen Anwendungen, wo Daten aus einer normalverteilten Grundgesamtheit stammen, aber keine ausreichenden Informationen über die Varianz vorliegen.
\item \textbf{Annahmen:} Die Stichproben müssen unabhängig und identisch verteilt sein, und die Grundgesamtheit muss normalverteilt sein. Der Test ist robust gegenüber Verletzungen der Normalverteilungsannahme, besonders bei größeren Stichproben.
\end{itemize}
\vspace{1cm}
\textbf{Zweistichproben-t-Test:}
\begin{itemize}
\item \textbf{Einsatzgebiet:} Wenn Sie die Mittelwerte von zwei unabhängigen Stichproben vergleichen möchten, wobei angenommen wird, dass beide Stichproben aus normalverteilten Grundgesamtheiten mit unbekannten, aber gleichen Varianzen stammen.
\item \textbf{Typische Situationen:} Dieser Test ist hilfreich, wenn Sie beispielsweise die Wirksamkeit zweier Medikamente oder die Unterschiede zwischen zwei verschiedenen Produktionsprozessen bewerten wollen.
\item \textbf{Annahmen:} Beide Stichproben müssen unabhängig voneinander sein, die Daten in beiden Gruppen müssen normalverteilt sein, und die Varianzen der beiden Grundgesamtheiten werden als gleich angenommen.
\end{itemize}

\subsection{Lösung 1}

Für das Szenario ist das Testverfahren 1 zu wählen, also einen Einstichproben-t-Test mit der Nullhypothese $H_0: \mu \leq 8,5$, da dies die vom Hersteller aufgestellte Behauptung ist, welche wir überprüfen wollen, wobei die Varianz unbekannt ist. 

\end{document}
