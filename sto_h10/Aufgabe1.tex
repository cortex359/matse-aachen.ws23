\documentclass[main.tex]{subfiles}

\begin{document}

\section{Aufgabe 1}
Der durchschnittliche Kraftstoffverbrauch (in Liter/100 km) von Autos eines bestimmten Typs kann durch eine $\mathcal{N}(\mu; \sigma^2)$-verteilte Zufallsvariable mit $\mu \geq 0$ und $\sigma > 0$ beschrieben werden. Der Hersteller gibt an, dass Fahrzeuge diesen Types einen Durchschnittsverbrauch von weniger als $8,5$ Liter/100 km haben. \\[2mm]
%
Entscheiden Sie, welches der folgenden Testverfahren zur Überprüfung der Aussage des Herstellers zu einem Signifikanzniveau $\alpha \in (0;1)$ geeignet ist. Begründen Sie ihre Antwort.
\begin{enumerate}
	\item[1)] der $t$-Test mit den Hypothesen $H_0: \mu \leq 8,5$ gegen $H_1: \mu > 8,5$
	\item[2)] der Gauß-Test mit den Hypothesen $H_0: \mu \geq 8,5$ gegen $H_1: \mu < 8,5$
	\item[3)] der $t$-Test mit den Hypothesen $H_0: \mu \geq 8,5$ gegen $H_1: \mu < 8,5$
	\item[4)] der $t$-Test mit den Hypothesen $H_0: \mu = 8,5$ gegen $H_1: \mu \neq 8,5$
\end{enumerate}

\subsection{Lösung 1}

\end{document}
