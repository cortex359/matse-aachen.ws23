\documentclass[main.tex]{subfiles}

\begin{document}

\section{Aufgabe 1}
Aus einer Urne, in der sich zwei Kugeln mit der Ziffer "`1"' und drei Kugeln mit der Ziffer "`2"' befinden, werden ohne Zurücklegen $2$ Kugeln entnommen. Es seien die folgenden Zufallsgrößen definiert
$$ X_1 = \{ \mbox{Ziffer der beim 1. Versuch gezogenen Kugel} \} $$
$$ X_2 = \{ \mbox{Ziffer der beim 2. Versuch gezogenen Kugel} \} $$
\begin{enumerate}
\item Bestimmen Sie die gemeinsame Wahrscheinlichkeitsfunktion $f(x_1, x_2)$.
\item Bestimmen Sie die Randverteilung der Zufallsvariablen $X_1$ und $X_2$.
\item Prüfen Sie die Zufallsvariablen auf stochastische Unabhängigkeit.
\end{enumerate}

\subsection{Lösung 1a}
Die gemeinsame Wahrscheinlichkeitsfunktion $P(x_1, x_2)$ ergibt sich über die Wahrscheinlichkeitstabelle der Zufallsvariablen $X_1, X_2$.
\begin{center}
    \begin{tabular}{*{4}{c|}}
        \multicolumn{2}{c}{} & \multicolumn{2}{c}{$X_1$} \\ \cline{3-4}
        \multicolumn{2}{c|}{} & 1 & 2 \\ \cline{2-4}
        \multirow{2}{1em}{$X_2$} & 1 & \sfrac{1}{10} & \sfrac{3}{10} \\ \cline{2-4}
                                 & 2 & \sfrac{3}{10} & \sfrac{3}{10} \\ \cline{2-4}
    \end{tabular}
\end{center}
Das bedeutet, wir können die Wahrscheinlichkeitsfunktion wie folgt angeben:
$$
    f(x_1, x_2) = \begin{cases}
        0,1 & \text{für } x_1 = x_2 = 1 \\
        0,3 & \text{für } x_1 = 2 \lor x_2 = 2 \\
        0 & \text{sonst}
    \end{cases}
$$

\subsection{Lösung 1b}
Randverteilung von $X_1$:
$$
f_1(x_1) = \begin{cases}
    0,4 & \text{für } x_1 = 1\\
    0,6 & \text{für } x_1 = 2\\
    0 & \text{sonst}
\end{cases}
$$

Randverteilung von $X_2$:
$$
f_2(x_2) = \begin{cases}
    0,4 & \text{für } x_2 = 1\\
    0,6 & \text{für } x_2 = 2\\
    0 & \text{sonst}
\end{cases}
$$

\subsection{Lösung 1c}
Die Zufallsvariablen sind stochastisch unabhängig, wenn für alle $x_1, x_2$ gilt: $$
    f(x_1, x_2) = f(x_1) \cdot f(x_2)
$$
Gegenbeispiel: $f(1, 1) = 0,1$ jedoch ist $f_1(1) \cdot f_2(1) = 0,4\cdot 0,4 = 0,16$. Somit sind die Zufallsvariablen stochastisch abhängig.


\end{document}
