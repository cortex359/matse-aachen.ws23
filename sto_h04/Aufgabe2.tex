\documentclass[main.tex]{subfiles}

\begin{document}

\section{Aufgabe 2}
$X$ repräsentiere die täglichen Verkäufe eines bestimmten Produktes und besitze die Wahrscheinlichkeitsverteilung:
\begin{center}
\begin{tabular}{c|c|c|c|c|c|c|c}
	$x$        & $7.000$ & $7.500$ & $8.000$ & $8.500$ & $9.000$ & $9.500$ & $10.000$ \\ 
	\hline
	$P(X = x)$ &  $0,05$ &   $0,2$ &  $0,35$ &  $0,19$ &  $0,12$ &  $0,08$ &   $0,01$ \\
\end{tabular}
\end{center}
\begin{enumerate}
\item Berechnen Sie  
\begin{enumerate}
  	\item den Erwartungswert,  
  	\item die Varianz sowie die Standardabweichung.
\end{enumerate}
\item Der Nettogewinn, der sich bei einem Verkauf von $X$ Einheiten ergibt, sei durch $Z = 5 \cdot X - 38.000$ gegeben. Berechnen Sie 
\begin{enumerate}
  	\item den Erwartungswert  
  	\item und die Varianz
\end{enumerate}
des Nettogewinns $Z$.
\end{enumerate}

\subsection{Lösung 2}

\end{document}
