\documentclass[main.tex]{subfiles}

\begin{document}

\section{Aufgabe 5}
Im letzten Wintersemester nahmen $120$ Studierende an der Stochastik-Klausur teil. Im Folgenden ist die Punkteverteilung einer Stochastik-Klausur-Aufgabe angegeben:
\begin{center}
\begin{tabular}{c|c|c|c|c|c|c}
	Punkte & $6$ & $5$ & $4$ & $3$ & $2$ & $1$ \\ \hline
	Anzahl & $0$ & $10$ & $30$ & $40$ & $20$ & $20$ \\ \hline
	Wahrscheinlichkeit & $0$ & $\sfrac{10}{120}$ & $\sfrac{30}{120}$ & $\sfrac{40}{120}$ & $\sfrac{20}{120}$ & $\sfrac{20}{120}$ \\
\end{tabular}
\end{center}
\begin{enumerate}
\item Berechnen Sie
\begin{enumerate}
	\item den Erwartungswert.
	\item die Varianz und die Standardabweichung.
\end{enumerate}
\item Mit welcher Wahrscheinlichkeit liegt der Punkteschnitt im Bereich $[\mu-2; \mu+2]$? Nutzen Sie zur Abschätzung die Tschebyscheffsche Ungleichung.
\end{enumerate}

\subsection{Lösung 5a}

Sei $X = \set{\text{Anzahl der Punkte}}$, so beträgt der Erwartungswert nach der gegebenen Tabelle $E(X) = \sfrac{35}{12} \approx 2,92.$
Um die Varianz zu ermitteln, berechnen wir zunächst $E(X^2) = \sfrac{119}{12} \approx 9,92$ und sodann $$\begin{aligned}
	\Var (X) &= E\left(X^2\right) - E(X)^2 \\
	&= \frac{119}{12} - \left(\frac{35}{12}\right)^2 \\
	&= \frac{203}{144} \approx 1,41.
\end{aligned}$$

Die Standardabweichung beträgt somit $\sigma = \sqrt{\frac{203}{144}} \approx 1,187.$

\subsection{Lösung 5b}
Nach Tschebyschow gilt allgemein $$\begin{aligned}
	P\left(\abs*{X-\mu} \geq k\right) &\leq \frac{\sigma^2}{k^2} \\
	P\left(\abs*{X-\mu} < k\right) &\geq 1 - \frac{\sigma^2}{k^2}.
\end{aligned}$$

Damit der Punkteschnitt im Bereich $[\mu-2; \mu+2]$ liegt, betrachten wir hier die zweite Gleichung und setzen $k=2$, da $P\left(X\in [\mu-2; \mu+2]\right) = P(\abs*{X-\mu \leq 2})$ ist.
\begin{equiveqs}
	& P\left(\abs*{X- \mu} < 2\right) &\geq 1 - \frac{\frac{203}{144}}{2^2} \\[3mm]
	\equiv & P\left(\abs*{X- \mu} < 2\right) &\geq \frac{373}{576} \approx 64,76\% \\
\end{equiveqs}

Mit Hilfe der Varianz $\Var = \sigma^2 \approx 1,187$ und dem Erwartungswert $\mu \approx 2,92$ lässt sich also sagen, dass
die Wahrscheinlichkeit, dass ein Studierender bei der Aufgabe zwischen 0,92 und 4,92 Punkte erzielen konnte, 64,76\% beträgt.

\end{document}
