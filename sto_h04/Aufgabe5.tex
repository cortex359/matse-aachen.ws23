\documentclass[main.tex]{subfiles}

\begin{document}

\section{Aufgabe 5}
Im letzten Wintersemester nahmen $120$ Studierende an der Stochastik-Klausur teil. Im Folgenden ist Punkteverteilung einer Stochastik-Klausur-Aufgabe angegeben:
\begin{center}
\begin{tabular}{|c|c|c|c|c|c|c|} \hline
	Punkte & $6$ & $5$ & $4$ & $3$ & $2$ & $1$ \\ \hline
	Anzahl & $0$ & $10$ & $30$ & $40$ & $20$ & $20$ \\ \hline
	Wahrscheinlichkeit & $0$ & $\frac{10}{120}$ & $\frac{30}{120}$ & $\frac{40}{120}$ & $\frac{20}{120}$ & $\frac{20}{120}$ \\ \hline
\end{tabular}
\end{center}
\begin{enumerate}
\item Berechnen Sie 
\begin{enumerate}
	\item den Erwartungswert.
	\item die Varianz und die Standardabweichung.
\end{enumerate}
\item Mit welcher Wahrscheinlichkeit liegt der Punkteschnitt im Bereich $[\mu-2; \mu+2]$? Nutzen Sie zur Abschätzung die Tschebyscheffsche Ungleichung.
\end{enumerate}

\subsection{Lösung 5}

\end{document}
