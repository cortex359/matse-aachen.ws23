\documentclass[main.tex]{subfiles}

\begin{document}

\section{Aufgabe 3}
Ein Taschenrechner liefert Zufallszahlen zwischen $0$ und $1$. Es wurden nacheinander $1.000$ dieser Zahlen erzeugt. Nach Einteilung des Intervalls $[0,1]$ in $10$ gleichgroße Teilintervalle wurde gezählt, wie viele der $1.000$ Zufallszahlen auf die einzelnen Klassen entfielen. Man erhielt folgende Tabelle:
\begin{center}
\begin{tabular}{|l|p{0.75cm}|p{0.75cm}|p{0.75cm}|p{0.75cm}|p{0.75cm}|p{0.75cm}|p{0.75cm}|p{0.75cm}|p{0.75cm}|p{0.75cm}|} \hline
Klasse & $[0,\newline0.1]$ & $(0,1; \newline 0,2]$ & $(0,2; \newline 0,3]$ 
	& $(0,3; \newline 0,4]$ & $(0,4; \newline 0,5]$ & $(0,5; \newline 0,6]$ 
	& $(0,6; \newline 0,7]$ & $(0,7; \newline 0,8]$ & $(0,8; \newline 0,9]$ 
	& $(0,9; \newline 1,0]$ \\ \hline
Anzahl & $68$ & $116$ & $101$ & $107$ & $92$ & $100$ & $136$ & $101$ & $79$ & $100$ \\ \hline
\end{tabular}
\end{center}
Mithilfe eines geeigneten Chi-Quadrat-Anpassungstests zum Niveau $\alpha=0,05$ überprüfe man, ob die Zufallszahlen $x_1,\dots,x_{1.000}$ als eine Folge von im Intervall $[0;1]$-gleichverteilten Zufallszahlen angesehen werden können.

\subsection{Lösung 3}

\end{document}
