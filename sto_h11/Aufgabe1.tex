\documentclass[main.tex]{subfiles}

\begin{document}

\section{Aufgabe 1}
In einem landwirtschaftlichen Betrieb erhielten von $20$ Versuchsrindern $10$ Rinder (Versuchsgruppe 1) jeden Tag Kraftfutter der Zusammensetzung $A$, die übrigen $10$ Rinder (Versuchsgruppe 2) erhielten das herkömmliche Futter der Zusammensetzung $B$. Nach einer gewissen Zeit wurde die Gewichtszunahme in $kg$ in beiden Gruppen festgestellt:
\begin{center}
	\begin{tabular}{ccccccccccc}
		Gruppe 1: & $7,2$ & $4,1$ & $5,5$ & $4,5$ & $5,7$ & $3,8$ & $4,6$ & $6,0$ & $5,2$ & $5,4$ \\
		Gruppe 2: & $5,3$ & $4,4$ & $5,0$ & $3,5$ & $3,9$ & $4,9$ & $5,6$ & $2,5$ & $4,0$ & $3,6$
	\end{tabular}
\end{center}
\begin{enumerate}
% a)
\item Unter der Annahme, dass sich die Gewichtszunahme durch unabhängige, in beiden Fällen identisch normalverteilte Zufallsvariablen beschreiben lässt, prüfe man mit einem geeigneten Test zum Niveau $\alpha = 0,1$, ob die Annahme, dass die Gewichtszunahme bei Verabreichung von Kraftfutter der Zusammensetzung $A$ die gleiche Streuung aufweist wie die Gewichtszunahme bei Verabreichung des herkömmlichen Futters der Zusammensetzung $B$, zu verwerfen ist.
% b)
\item Unter der Annahme, dass sich die Gewichtszunahme durch unabhängige, in beiden Fällen identisch normalverteilte Zufallsvariablen mit gleicher Varianz beschreiben lässt, prüfe man mit einem geeigneten Test zum Niveau $\alpha = 0,025$ die Hypothese, dass die Gewichtszunahme bei Verabreichung von Kraftfutter der Zusammensetzung $A$ nicht größer ist als die Gewichtszunahme bei Verabreichung des herkömmlichen Futters der Zusammensetzung $B$.
\end{enumerate}

\subsection{Lösung 1}

Sei $X =\set{\text{Gewichtszunahme in kg}} \sim \mathcal{N}$ und $\mu, \sigma^2$ unbekannt.
Es soll zum Signifikanzniveau $\alpha = 0,1$ geprüft werden, ob
$$
H_0: \sigma_1^2 = \sigma_2^2 \qquad \text{gegen} \qquad H_1: \sigma_1^2 \neq \sigma_2^2
$$
Bestand hat.

Wir bestimmen den empirischen Mittelwert $\overline{x}_A = \frac{52}{10} = 5,2$ kg von Gruppe 1 und $\overline{x}_B = \frac{42,7}{10} = 4,27$ kg von Gruppe 2, sowie die korrigierte empirische Varianz:
$$\begin{aligned}
	s^2 &= \frac{1}{n-1} \sum_{i=1}^n (x_i - \overline{x})^2 \\
	&= \frac{1}{n-1} \left( \sum_{i=1}^n \left( x_i^2 \right) - n \cdot \overline{x}^2 \right) \\
\end{aligned}$$

Also für Gruppe 1:
$$\begin{aligned}
	s_A^2 &= \frac{1}{9} \left( 279,44 - 10 \cdot 27,04 \right) \\
	&= 1,00\overline{4} \\
\end{aligned}$$
und für Gruppe 2:
$$\begin{aligned}
	s_B^2 &= \frac{1}{9} \left( 190,49 - 10 \cdot 18,2329 \right) \\
	&= 0,906\overline{7} \\
\end{aligned}$$

Unter der Hypothese $H_0$ liegt eine F-Verteilung
$$
	\frac{s_A^2}{s_B^2} \underset{H_0}{\sim} F_{n-1,\ m-1}
$$
mit 9 Freiheitsgraden vor.

Test:
$$\begin{aligned}
	\frac{s_A^2}{s_B^2} < F_{9;\ 9;\ \frac{\alpha}{2}}
	\quad &\lor \quad
	\frac{s_A^2}{s_B^2} > F_{9;\ 9;\ \left( 1 - \frac{\alpha}{2}\right)}\\
%
	1,107707 < F_{9;\ 9;\ 0,05}
	\quad &\lor \quad
	1,107707 > F_{9;\ 9;\ 0,95}\\
\end{aligned}$$

Da $F_{9;\ 9;\ 0,05} = 0,314575$ und $F_{9;\ 9;\ 0,95} = 3,17889$ (unteres/oberes 5\%-Quantil der $F_{9;9}$-Verteilung) also
$$
	0,314575 \leq 1,107707 \leq 3,17889
$$
ist, wird $H_0$ nicht abgelehnt.

\end{document}
