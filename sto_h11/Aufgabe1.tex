\documentclass[main.tex]{subfiles}

\begin{document}

\section{Aufgabe 1}
In einem landwirtschaftlichen Betrieb erhielten von $20$ Versuchsrindern $10$ Rinder (Versuchsgruppe 1) jeden Tag Kraftfutter der Zusammensetzung $A$, die übrigen $10$ Rinder (Versuchsgruppe 2) erhielten das herkömmliche Futter der Zusammensetzung $B$. Nach einer gewissen Zeit wurde die Gewichtszunahme in $kg$ in beiden Gruppen festgestellt:
\begin{center}
	\begin{tabular}{ccccccccccc}
		Gruppe 1: & $7,2$ & $4,1$ & $5,5$ & $4,5$ & $5,7$ & $3,8$ & $4,6$ & $6,0$ & $5,2$ & $5,4$ \\
		Gruppe 2: & $5,3$ & $4,4$ & $5,0$ & $3,5$ & $3,9$ & $4,9$ & $5,6$ & $2,5$ & $4,0$ & $3,6$ 
	\end{tabular}
\end{center}
\begin{enumerate}
% a)
\item Unter der Annahme, dass sich die Gewichtszunahme durch unabhängige, in beiden Fällen identisch normalverteilte Zufallsvariablen beschreiben lässt, prüfe man mit einem geeigneten Test zum Niveau $\alpha = 0,1$, ob die Annahme, dass die Gewichtszunahme bei Verabreichung von Kraftfutter der Zusammensetzung $A$ die gleiche Streuung aufweist wie die Gewichtszunahme bei Verabreichung des herkömmlichen Futters der Zusammensetzung $B$, zu verwerfen ist.
% b)
\item Unter der Annahme, dass sich die Gewichtszunahme durch unabhängige, in beiden Fällen identisch normalverteilte Zufallsvariablen mit gleicher Varianz beschreiben lässt, prüfe man mit einem geeigneten Test zum Niveau $\alpha = 0,025$ die Hypothese, dass die Gewichtszunahme bei Verabreichung von Kraftfutter der Zusammensetzung $A$ nicht größer ist als die Gewichtszunahme bei Verabreichung des herkömmlichen Futters der Zusammensetzung $B$.
\end{enumerate}

\subsection{Lösung 1}

\end{document}
